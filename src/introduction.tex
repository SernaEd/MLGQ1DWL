Quantum dispersion in graphene, understood as the relationship between energy levels and their respective wave vectors, is essential for explaining the outstanding electronic and transport properties observed in this unique two-dimensional material.
Its hexagonal lattice structure generates massless Dirac fermions, which gives rise to particular quantum phenomena such as extremely high mobility, ballistic transport at micrometric scales, and intrinsic absence of a band gap\cite{Geimk2007}.
Among these phenomena, it is especially interesting to study the role of SOIR, which can significantly modify electronic behavior and lead to novel interference and scattering effects related to electron spin.

A central theme of this work is the intertwined role of pseudospin (sublattice), real spin, and carrier momentum in graphene.
Their coupling—through chirality in pristine graphene and through SOIR in proximitized or field-engineered systems—governs boundary matching, mode mixing, and transmission across electrostatic profiles.
This interrelation can drive counterintuitive transport regimes, including the suppression of the normally perfect Klein transmission (anti-Klein tunneling) under suitable Rashba strengths, confinement conditions, and barrier parameters\cite{Young2009, DellAnnaJPhysCondMatt2018}.
In what follows, we make this connection explicit by analyzing how the effective Rashba coupling within the barrier reshapes wave-packet dynamics and transmission resonances.
The precise understanding of these interactions is critical not only from a theoretical perspective but also represents the fundamental basis for emerging technological applications in strategic areas such as spintronics and quantum computing\cite{WeizheMaterials2017, AvsarNatCommun2014, LiuNano2023}.
In this context, the eigenvalue equation provides the fundamental framework for analyzing electronic transport, with its numerical solution being particularly suitable through robust methods such as the finite difference technique (FDM)\cite{Carrillo2015}.
Consequently, the central objective of this work is to calculate in detail the energy dispersion relations for electrons in monolayer graphene under SOIR interaction using Gaussian wave packets, highlighting how these quantum effects can be exploited for the advanced engineering of nanotechnological devices specialized in spintronic and quantum information applications.

It is equally important to emphasize the relevance of SOIR as a general phenomenon that manifests well beyond graphene.
SOIR arises from structural inversion asymmetry and appears across diverse platforms, including semiconductor two-dimensional electron gases, oxide interfaces, topological-insulator surfaces, transition-metal dichalcogenides, heavy-metal/ferromagnet bilayers, and van der Waals heterostructures.
Emphasizing this generality clarifies that the mechanisms discussed here—spin splitting, spin–momentum locking, modified transmission resonances, and Klein-to–anti-Klein crossovers—have broader significance and can inform the design of spintronic and quantum devices across these material families\cite{ManchonNatureMater2015}.

We also need to to clarify our modeling scope regarding Rashba spin–orbit coupling in graphene.

In pristine monolayer graphene, the canonical (Dirac–Rashba) term is typically very small (often quoted in the $\mu$eV range), and thus it is commonly neglected for transport at the energy and length scales considered here.

This is the rationale behind not including the standard Dirac–Rashba term explicitly in the pristine regions of our model.
However, the literature also reports proximity-induced scenarios where a giant Rashba coupling can emerge, for example in graphene interfaced with transition-metal dichalcogenides or in systems with Au adatoms/intercalation, leading to effective SOC values in the meV–tens of meV range\cite{AvsarNatCommun2014, WangPhysRevX2016}.

In such cases, the canonical Dirac–Rashba term should be included and its impact on wave-packet propagation analyzed (e.g., stronger spin splitting, spin precession, modified transmission resonances, and even transitions from Klein to anti-Klein regimes).

In the present study we focus on the pristine-graphene limit where the canonical term is negligible, while we model proximity-induced effects within the barrier via an effective Rashba-like coupling, as detailed in Sec.~\ref{subsec:mathematical-model}.


Finally, it is worth noting that one of the essential aspects addressed in this work lies in the precise analytical formulation of the current vector, whose analytical expression is given by equation\eqref{eq:componentes}.
This expression is the standard charge-current operator associated with the Dirac–Rashba model in graphene and has been discussed in prior literature\cite{AvishaiPhysRevB2021}.
Our contribution is to employ this established result to perform a component-wise and correlation-based analysis tailored to time-resolved Gaussian wave-packet transport under SOIR, which provides practical interpretability of the numerical transmission features reported in this study.


%\subsection{Weak Localization}\label{subsec:weak-localization}
%
%        Weak localization (WL) and weak antilocalization (WAL) are quantum phenomena arising from the interference of electron waves in disordered materials.
%         In graphene, these effects are heavily influenced by spin-orbit interactions (SOI), which describe the coupling between an electron's spin and its momentum.
%          Strong SOI can cause a transition from WL to WAL, modifying the phase coherence length (the distance over which electron waves maintain their phase relationship) and impacting electrical conductivity\cite{WeizheMaterials2017,AvsarNatCommun2014}.
%           This transition is crucial for understanding spin interactions within graphene and designing quantum devices.

\subsection{Klein Paradox}\label{subsec:kleinParadox}

Applying an electrostatic potential to graphene can create regions where electrons are classically forbidden due to energy constraints\cite{SoninPhysRevB2009}.
For instance, a gate voltage can shift the local band structure (i.e., the Dirac point) relative to the Fermi level, creating a p-type region in which, at the incident electron energy, only valence-band states are available; classically, electrons lack sufficient energy to surmount this potential barrier\cite{DellAnnaJPhysCondMatt2018}.
However, in graphene, electron transmission through such barriers can exhibit unusual behavior due to the Klein paradox\cite{Young2009}.

The Klein paradox describes the phenomenon where electrons in graphene can tunnel through electrostatic potential barriers with a probability approaching unity, even at perpendicular incidence, seemingly defying the principles of classical mechanics\cite{TrauzettelNature2007}.
This counterintuitive transmission arises from the relativistic nature of electrons in graphene, where, within the barrier region, incident electrons can be converted into holes, allowing for unimpeded passage\cite{BernardiniJPhysAMathTheor2010}.
The Klein paradox is a direct consequence of graphene's unique linear dispersion relation and the associated massless Dirac fermion behavior.
While this perfect transmission through barriers might pose challenges for achieving electron confinement in some device designs, it also presents opportunities for developing novel tunneling-based electronic devices.
Beyond simply creating a forbidden region, the application of electrostatic potential can dramatically alter electron transmission through various other mechanisms, influencing the refractive index and leading to phenomena like electron lensing\cite{ParedesPhysRevB2021}.

\subsection{Rashba-Type Spin-Orbit Interaction}\label{subsec:rashba-type-spin-orbit-interaction}

SOIR is a relativistic effect that arises in systems lacking structural inversion symmetry\cite{AvishaiPhysRevB2021}.
SOIR is especially relevant due to its ability to significantly modify the electronic structure of graphene, inducing particular couplings between the spin and momentum of electrons.
This leads to emerging phenomena such as spin-dependent quantum interference and unconventional scattering effects.
The impact of SOIR not only provides new perspectives in fundamental physics but also opens promising possibilities for advanced applications in spintronic technologies and quantum information-based devices.

In graphene, SOIR can be induced by an external electric field applied perpendicular to the graphene plane or through proximity to a substrate\cite{ShcherbakovSciAdv2021}.
A key consequence of SOIR is the breaking of spin degeneracy, where the energy levels of electrons with opposite spins are no longer the same, even in the absence of an external magnetic field\cite{DelkhoshPhysE2015}.
Controlling spin degeneracy through electric fields via SOIR is particularly valuable for spintronic applications.
It enables electron spin manipulation without depending on magnetic fields, which typically require more power and are challenging to incorporate effectively into nanoscale systems.


Furthermore, SOIR can strongly reshape the band dispersion and spin textures in graphene and related heterostructures\cite{WangPhysRevX2016}.
In proximity-engineered systems that additionally break sublattice symmetry or combine multiple SOC channels, gaps and effective-mass renormalizations may emerge at low energies\cite{WangPhysRevX2016, AvsarNatCommun2014}.
In materials with strong spin–orbit coupling (e.g., heavy-hole systems), higher-order variants such as $k^3$ Rashba terms produce anisotropic bands and additional spin-degeneracy points under illumination\cite{DellAnnaJPhysCondMatt2018, AvishaiPhysRevB2021}.
While that specific example pertains to heavy holes, the broader principle—that SOIR can markedly alter dispersion and transport—remains relevant to graphene, especially under proximity to high-SOC substrates\cite{GindikinPhysRevB2025}.
These dispersion changes can significantly impact mobility and the density of states.
Notably, tuning SOIR in graphene can drive crossovers between Klein and anti-Klein tunneling across electrostatic barriers\cite{DellAnnaJPhysCondMatt2018}.
This control of barrier transmission via SOIR underpins opportunities for spintronic and quantum devices\cite{YaoMater2024}.

In the next section, the physical-mathematical model adopted in this study will be described in detail.
Building on the conceptual foundation presented earlier regarding quantum dispersion and Rashba-type spin-orbit interaction, the system and corresponding equations that allow numerical simulation of the phenomena observed in our results will be defined.


%    \subsection{Quantum Dispersion and Anomalous Hall Effect}\label{subsec:quantum-dispersion-and-anomalous-hall-effect}
%
%    These interactions also impact quantum dispersion, the relationship between an electron's energy and momentum.
%    The anomalous Hall effect (AHE), where a transverse voltage arises without an external magnetic field, has been observed in edge-bonded monolayer graphene\cite{LiuNano2023}.
%     This demonstrates both ordinary and anomalous Hall effects, indicating long-range ferromagnetic order (spontaneous alignment of electron spins) and suggesting applications in carbon-based spintronics\cite{YaoMater2024}.
%      The theoretical prediction of the quantum anomalous Hall (QAH) effect in graphene-based heterostructures further highlights the potential for hosting non-trivial topological phases.
%
%
%    Electrostatic potential and Rashba spin-orbit interaction dramatically alter electron transmission in graphene, even at perpendicular incidence, defying the Klein paradox.
%
%This results from the potential generating a classically forbidden region, SOIR breaking spin degeneracy and modifying the heavy hole's effective mass, spin-dependent scattering at interfaces, the emergence of quasi-bound states that enable resonant tunneling, and the heavy hole's effective mass playing a crucial role in tunneling dynamics.
%
%      These factors, particularly the deviation from the Dirac point dispersion relation, significantly modify transmission, requiring further computational and experimental investigation to fully elucidate this phenomenon and refine our understanding of graphene electron transport.
