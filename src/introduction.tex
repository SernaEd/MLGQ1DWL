La dispersión cuántica en grafeno, entendida como la relación entre niveles de energía y sus respectivos vectores de onda, es esencial para explicar las sobresalientes propiedades electrónicas y de transporte observadas en este material bidimensional único.
Su estructura de red hexagonal genera fermiones de Dirac sin masa, lo que origina particulares fenómenos cuánticos como movilidad extremadamente alta, transporte balístico en escalas micrométricas, y ausencia intrínseca de banda prohibida\cite{Serna2019}.
Dentro de estos fenómenos, resulta especialmente interesante estudiar el papel de la SOIR, que puede modificar significativamente el comportamiento electrónico y conducir a efectos novedosos de interferencia y dispersión relacionados con el espín electrónico.

La comprensión precisa de estas interacciones es crítica no solamente desde la perspectiva teórica, sino que además representa la base fundamental para aplicaciones tecnológicas emergentes en áreas estratégicas tales como la spintrónica y la computación cuántica\cite{WeizheMaterials2017, AvsarNatCommun2014, LiuNano2023}.
En este contexto, la ecuación de Schrödinger provee el marco fundamental para analizar el transporte electrónico, siendo particularmente idónea su solución numérica mediante métodos robustos como la técnica de diferencias finitas (FDM).
Consecuentemente, el objetivo central de este trabajo es calcular detalladamente las relaciones de dispersión energética para electrones en grafeno monocapa bajo interacción SOIR utilizando paquetes de onda gaussianos, destacando cómo estos efectos cuánticos pueden ser explotados para la ingeniería avanzada de dispositivos nanotecnológicos especializados en aplicaciones spintrónicas y de información cuántica.


Finalmente, cabe destacar que uno de los aspectos esenciales abordados en este trabajo radica en la formulación analítica precisa del vector corriente, cuya expresión analítica es dada por la ecuación\eqref{eq:componentes}.
Esta expresión representa un pilar fundamental para comprender detalladamente los efectos y mecanismos subyacentes en los fenómenos de transporte electrónico en presencia de SOIR\@.
Por ello, la obtención y análisis de dicha ecuación constituye una aportación significativa al campo, especialmente en aplicaciones relacionadas con dispositivos spintrónicos y tecnologías cuánticas emergentes.


%\subsection{Weak Localization}\label{subsec:weak-localization}
%
%        Weak localization (WL) and weak antilocalization (WAL) are quantum phenomena arising from the interference of electron waves in disordered materials.
%         In graphene, these effects are heavily influenced by spin-orbit interactions (SOI), which describe the coupling between an electron's spin and its momentum.
%          Strong SOI can cause a transition from WL to WAL, modifying the phase coherence length (the distance over which electron waves maintain their phase relationship) and impacting electrical conductivity\cite{WeizheMaterials2017,AvsarNatCommun2014}.
%           This transition is crucial for understanding spin interactions within graphene and designing quantum devices.

\subsection{Klein Paradox}\label{subsec:kleinParadox}

Applying an electrostatic potential to graphene can create regions where electrons are classically forbidden due to energy constraints\cite{SoninPhysRevB2009}.
For instance, a negative gate voltage can shift the Fermi level into the valence band, establishing a potential barrier that electrons do not have sufficient energy to overcome, according to classical physics\cite{DellAnnaJPhysCondMatt2018}.
However, in graphene, electron transmission through such barriers can exhibit unusual behavior due to the Klein paradox.

The Klein paradox describes the phenomenon where electrons in graphene can tunnel through electrostatic potential barriers with a probability approaching unity, even at perpendicular incidence, seemingly defying the principles of classical mechanics\cite{TrauzettelNature2007}.
This counterintuitive transmission arises from the relativistic nature of electrons in graphene, where, within the barrier region, incident electrons can be converted into holes, allowing for unimpeded passage\cite{BernardiniJPhysAMathTheor2010}.
The Klein paradox is a direct consequence of graphene's unique linear dispersion relation and the associated massless Dirac fermion behavior.
While this perfect transmission through barriers might pose challenges for achieving electron confinement in some device designs, it also presents opportunities for developing novel tunneling-based electronic devices.
Beyond simply creating a forbidden region, the application of electrostatic potential can dramatically alter electron transmission through various other mechanisms, influencing the refractive index and leading to phenomena like electron lensing\cite{ParedesPhysRevB2021}.

\subsection{Rashba-Type Spin-Orbit Interaction}\label{subsec:rashba-type-spin-orbit-interaction}

SOIR is a relativistic effect that arises in systems lacking structural inversion symmetry\cite{AvishaiPhysRevB2021}.
La SOIR es especialmente relevante debido a su capacidad para modificar significativamente la estructura electrónica del grafeno, induciendo acoplamientos particulares entre el espín y el momento de los electrones.
Esto lleva a exhibir fenómenos emergentes tales como la interferencia cuántica dependiente del espín y efectos de dispersión no convencionales.
El impacto de la SOIR no solo aporta nuevas perspectivas en física fundamental, sino que también abre posibilidades prometedoras para aplicaciones avanzadas en tecnologías spintrónicas y dispositivos basados en información cuántica.

In graphene, SOIR can be induced by an external electric field applied perpendicular to the graphene plane or through proximity to a substrate\cite{ShcherbakovSciAdv2021}.
A key consequence of SOIR is the breaking of spin degeneracy, where the energy levels of electrons with opposite spins are no longer the same, even in the absence of an external magnetic field\cite{DelkhoshPhysE2015}.
Controlling spin degeneracy through electric fields via SOIR is particularly valuable for spintronic applications.
It enables electron spin manipulation without depending on magnetic fields, which typically require more power and are challenging to incorporate effectively into nanoscale systems.


Furthermore, SOIR modifies the effective mass of charge carriers in graphene\cite{WangPhysRevX2016}.
It can lead to the opening of an energy gap at the Dirac points, effectively giving the otherwise massless Dirac fermions a finite mass.
In systems with strong spin-orbit coupling, such as certain heavy-hole systems, specific types of SOIR (like $k^3$-Rashba spin-orbit coupling) can even result in anisotropic band structures and the emergence of additional spin-degeneracy points under light illumination.
While this specific example refers to heavy holes, the principle that SOIR can significantly alter the band structure and effective mass of carriers is relevant to graphene, particularly when considering proximity-induced effects from substrates with strong spin-orbit coupling\cite{GindikinPhysRevB2025}.
The modification of the effective mass by SOIR can profoundly impact graphene's transport properties, influencing carrier mobility and the density of states.
Notably, tuning the strength of SOIR in graphene can even induce a transition from the Klein tunneling regime (perfect transmission) to the anti-Klein tunneling regime (perfect reflection) when electrons encounter a potential barrier.
This ability to control the transmission probability through potential barriers using SOIR opens up exciting possibilities for creating novel electronic devices that leverage this fundamental quantum mechanical phenomenon\cite{YaoMater2024}.

En la próxima sección, se describirá detalladamente el modelo físico-matemático adoptado en este estudio.
Partiendo de la base conceptual presentada anteriormente sobre la dispersión cuántica y la interacción espín-órbita tipo Rashba, se definirá el sistema y las ecuaciones correspondientes que permiten simular numéricamente los fenómenos observados en nuestros resultados.


%    \subsection{Quantum Dispersion and Anomalous Hall Effect}\label{subsec:quantum-dispersion-and-anomalous-hall-effect}
%
%    These interactions also impact quantum dispersion, the relationship between an electron's energy and momentum.
%    The anomalous Hall effect (AHE), where a transverse voltage arises without an external magnetic field, has been observed in edge-bonded monolayer graphene\cite{LiuNano2023}.
%     This demonstrates both ordinary and anomalous Hall effects, indicating long-range ferromagnetic order (spontaneous alignment of electron spins) and suggesting applications in carbon-based spintronics\cite{YaoMater2024}.
%      The theoretical prediction of the quantum anomalous Hall (QAH) effect in graphene-based heterostructures further highlights the potential for hosting non-trivial topological phases.
%
%
%    Electrostatic potential and Rashba spin-orbit interaction dramatically alter electron transmission in graphene, even at perpendicular incidence, defying the Klein paradox.
%
%This results from the potential generating a classically forbidden region, SOIR breaking spin degeneracy and modifying the heavy hole's effective mass, spin-dependent scattering at interfaces, the emergence of quasi-bound states that enable resonant tunneling, and the heavy hole's effective mass playing a crucial role in tunneling dynamics.
%
%      These factors, particularly the deviation from the Dirac point dispersion relation, significantly modify transmission, requiring further computational and experimental investigation to fully elucidate this phenomenon and refine our understanding of graphene electron transport.