%! Author = Eduardo Serna

% Preamble
\documentclass[twocolumn, aps, PRB, 10pt]{revtex4-2}

% Packages
\usepackage{amsmath}
\usepackage{graphicx} % For including images
\usepackage{hyperref} % For hyperlinks
\usepackage{amsfonts}
\usepackage{mathtools}
\usepackage{microtype}
\setlength{\emergencystretch}{2em}
\usepackage[utf8]{inputenc}
\usepackage[T1]{fontenc}

\newcommand{\bn}[1]{\mbox{\boldmath $#1$}}
\newcommand{\mb}{\mbox}
\newcommand{\angstrom}{\textup{\AA}}

% Document
\begin{document}

% Title Section
    \title{Weak Localization in Mono-layer Graphene with Rashba Spin-Orbit Interaction}
    \author{Eduardo Serna}
    \email{sernaed95@gmail.com}
    \affiliation{Centro de Investigación en Ciencias, Universidad Autónoma del Estado de Morelos, Morelos, 62209, México}.
    \author{I. Rodríguez Vargas}
    \email{isaac@fisica.uaz.edu.mx}
    \affiliation{Unidad Académica de Física, Universidad Autónoma de Zacatecas, Zacatecas, 98060, México}.
    \author{L. Diago-Cisneros}
    \email{ldiago@fisica.uh.cu}
    \affiliation{Facultad de Física, Universidad de La Habana, La Habana, 10400, Cuba}.
    \date{26 December 2024}


% Abstract
    \begin{abstract}
        This paper presents a theoretical and computational study on the impact of Rashba spin-orbit interaction (SOIR) on electron transmission through a potential barrier in monolayer graphene.
        The energy dispersion relations were numerically obtained via the finite differences method applied to the Schrödinger equation, using a model Hamiltonian incorporating graphene, SOIR, and a potential barrier.
        Our findings reveal that the presence of SOIR significantly modifies the electron transmission coefficients compared with pristine graphene conditions.
        While electron transmission in pristine graphene remains near unity with minor reductions due to quantum interference effects in a Gaussian wave packet, introducing SOIR leads to intricate transmission patterns dependent on the barrier parameters and wave packet characteristics.
        These effects arise from phenomena such as spin degeneracy lifting and spin-dependent scattering mechanisms.
        The results emphasize the critical role that SOIR plays in engineering sophisticated quantum devices, offering potential applications in spintronic technologies and quantum computing.
    \end{abstract}

    \maketitle

% Table of Contents (optional)
%    \tableofcontents
%    \newpage

% Introduction
    \section{Introduction}\label{sec:introduction}
    % Introduction: Quantum Dispersion and Dirac Fermion Tunneling in Graphene

Graphene exhibits a linear dispersion near the Dirac points, leading to massless Dirac-fermion behavior.
The phenomenon of Klein tunneling—perfect transmission of carriers at normal incidence through high potential barriers—is a hallmark of such systems.
Conversely, bilayer graphene shows anti-Klein tunneling, i.e., total reflection at normal incidence in the absence of a bandgap.
Over the past decade, both theoretical predictions and experimental validations of these phenomena have matured significantly, particularly in monolayer and few-layer graphene systems.

In pristine monolayer graphene, the canonical (Dirac–Rashba) term is typically very small (often quoted in the $\mu$eV range), and thus it is commonly neglected for transport at the energy and length scales considered here.
This is the rationale behind not including of the standard Dirac–Rashba term explicitly in the pristine regions of our model.
However, the literature also reports proximity-induced scenarios where a giant Rashba coupling can emerge.
For example, in graphene interfaced with transition-metal dichalcogenides or in systems with Au adatoms/intercalation, leading to effective SOC values in the meV–tens of meV range\cite{AvsarNatCommun2014, WangPhysRevX2016}.

In such cases, the canonical Dirac–Rashba term should be included and its impact on wave-packet propagation analyzed (e.g., stronger spin splitting, spin precession, modified transmission resonances, and even transitions from Klein to anti-Klein regimes).
In the present study we focus on the pristine-graphene limit where the canonical term is negligible, while we model proximity-induced effects within the barrier via an effective Rashba-like coupling, as detailed in Sec.~\ref{subsec:mathematical-model}.

Finally, it is worth noting that one of the essential aspects addressed in this work lies in the precise analytical formulation of the current vector, whose analytical expression is given by equation\eqref{eq:componentes}.
This expression is the standard charge-current operator associated with the Dirac–Rashba model in graphene and has been discussed in prior literature\cite{AvishaiPhysRevB2021}.
Our contribution is to employ this established result to perform a component-wise and correlation-based analysis tailored to time-resolved Gaussian wave-packet transport under SOIR, which provides practical interpretability of the numerical transmission features reported in this study.


\subsection{Klein Paradox}\label{subsec:kleinParadox}

Applying an electrostatic potential to graphene can create regions where electrons are classically forbidden due to energy constraints\cite{SoninPhysRevB2009}.
For instance, a gate voltage can shift the local band structure (i.e., the Dirac point) relative to the Fermi level, creating a p-type region in which, at the incident electron energy, only valence-band states are available; classically, electrons lack sufficient energy to surmount this potential barrier\cite{DellAnnaJPhysCondMatt2018}.
However, in graphene, electron transmission through such barriers can exhibit unusual behavior due to the Klein paradox\cite{Young2009}.

The Klein paradox describes the phenomenon where electrons in graphene can tunnel through electrostatic potential barriers with a probability approaching unity, even at perpendicular incidence, seemingly defying the principles of classical mechanics\cite{TrauzettelNature2007}.
This counterintuitive transmission arises from the relativistic nature of electrons in graphene, where, within the barrier region, incident electrons can be converted into holes, allowing for unimpeded passage\cite{BernardiniJPhysAMathTheor2010}.
The Klein paradox is a direct consequence of graphene's unique linear dispersion relation and the associated massless Dirac fermion behavior.
While this perfect transmission through barriers might pose challenges for achieving electron confinement in some device designs, it also presents opportunities for developing novel tunneling-based electronic devices.
Beyond simply creating a forbidden region, the application of electrostatic potential can dramatically alter electron transmission through various other mechanisms, influencing the refractive index and leading to phenomena like electron lensing\cite{ParedesPhysRevB2021}.

\subsection{Rashba-Type Spin-Orbit Interaction}\label{subsec:rashba-type-spin-orbit-interaction}

SOIR is a relativistic effect that arises in systems lacking structural inversion symmetry\cite{AvishaiPhysRevB2021}.
SOIR is especially relevant due to its ability to significantly modify the electronic structure of graphene, inducing particular couplings between the spin and momentum of electrons.
This leads to emerging phenomena such as spin-dependent quantum interference and unconventional scattering effects.
The impact of SOIR not only provides new perspectives in fundamental physics but also opens promising possibilities for advanced applications in spintronic technologies and quantum information-based devices.

In graphene, SOIR can be induced by an external electric field applied perpendicular to the graphene plane or through proximity to a substrate\cite{ShcherbakovSciAdv2021}.
A key consequence of SOIR is the breaking of spin degeneracy, where the energy levels of electrons with opposite spins are no longer the same, even in the absence of an external magnetic field\cite{DelkhoshPhysE2015}.
Controlling spin degeneracy through electric fields via SOIR is particularly valuable for spintronic applications.
It enables electron spin manipulation without depending on magnetic fields, which typically require more power and are challenging to incorporate effectively into nanoscale systems.


Furthermore, SOIR can strongly reshape the band dispersion and spin textures in graphene and related heterostructures\cite{WangPhysRevX2016}.
In proximity-engineered systems that additionally break sublattice symmetry or combine multiple SOC channels, gaps and effective-mass renormalizations may emerge at low energies\cite{WangPhysRevX2016, AvsarNatCommun2014}.
In materials with strong spin–orbit coupling (e.g., heavy-hole systems), higher-order variants such as $k^3$ Rashba terms produce anisotropic bands and additional spin-degeneracy points under illumination\cite{DellAnnaJPhysCondMatt2018, AvishaiPhysRevB2021}.
While that specific example pertains to heavy holes, the broader principle—that SOIR can markedly alter dispersion and transport—remains relevant to graphene, especially under proximity to high-SOC substrates\cite{GindikinPhysRevB2025}.
These dispersion changes can significantly impact mobility and the density of states.
Notably, tuning SOIR in graphene can drive crossovers between Klein and anti-Klein tunneling across electrostatic barriers\cite{DellAnnaJPhysCondMatt2018}.
This control of barrier transmission via SOIR underpins opportunities for spintronic and quantum devices\cite{YaoMater2024}.

\subsection{State of the Art}\label{subsec:state-of-the-art}

The theoretical foundations of Dirac fermion tunneling in graphene have been extensively developed over the past decade.
Early theoretical work established that in monolayer graphene, carriers preserve pseudospin alignment, leading to unit transmission at normal incidence regardless of barrier height\cite{Chen2016}.
In bilayer graphene, the Berry phase differs, resulting in anti-Klein tunneling where normal incidence yields zero transmission\cite{Du2018}.
Further theoretical contributions revealed the potential for tuning between anti-Klein and Klein regimes in bilayer graphene by introducing a tunable bandgap via perpendicular electric fields\cite{Du2018}.
Electron-optics analogies, such as negative refraction and Veselago lenses, were proposed, and models incorporating smooth versus abrupt barriers, superlattices, and periodic scatterers were developed to describe how pseudospin conservation can be manipulated or suppressed to control tunneling\cite{Walls2015,An2020}.

Experimental verification of these theoretical predictions has provided compelling evidence for the unique transport properties of graphene.
Electron-optics experiments confirmed negative refraction and angular-dependent transmission using ballistic graphene p–n junctions in encapsulated devices\cite{Chen2016}.
STM studies demonstrated quasi-bound states in nanometric graphene quantum dots, evidencing partial confinement of Dirac fermions despite Klein tunneling\cite{Gutierrez2016}.
Visualization of electron flow through a Veselago lens using scanning gate microscopy provided direct imaging of electron focusing effects\cite{Brun2019}.
In bilayer graphene Fabry–Pérot interferometers, tuning the Fermi energy allowed observation of Berry phase evolution from $2\pi$ to $\pi$, accompanied by the transition from anti-Klein to Klein tunneling regimes\cite{Du2018}.
The clearest demonstration of both Klein and anti-Klein tunneling in a single experimental platform was achieved via a Corbino geometry device, establishing angle-resolved transmission characteristics consistent with theoretical predictions\cite{Elahi2024}.

Device applications leveraging these fundamental phenomena have shown significant promise.
Klein tunnel field-effect transistors (GKTFETs) exploit angular filtering using crossed p–n junctions to achieve current modulation without a bandgap\cite{Tan2017}.
Theoretical models forecast ON/OFF ratios above $10^4$ and steep subthreshold slopes under ideal conditions.
Experimental prototypes yielded ON/OFF ratios of approximately 10 and demonstrated current saturation and enhanced output resistance, indicating potential for high-frequency analog applications\cite{Wang2019}.
Valley-selective Klein tunneling via superlattice barriers has been proposed as a route to valleytronic devices, though experimental realization remains pending\cite{An2020}.


\subsection{Critical analysis and opportunities}\label{subsec:future-work}

Despite significant advances in both theoretical understanding and experimental validation, several critical challenges and opportunities remain for future research in graphene-based quantum transport.

Klein tunnel transistors exhibit limited ON/OFF contrast due to edge scattering; strategies such as edgeless geometries or improved gating precision could address this limitation.
Valley-filtering mechanisms, while theoretically robust, lack experimental validation; designing patterned substrates or moiré structures may provide a viable path forward.
The exploration of trilayer graphene in terms of intermediate Berry phases and angular transmission profiles remains largely unexplored territory with significant potential.

Integration of electron-optics elements such as lenses and collimators in circuit-level architectures may unlock novel functional devices that leverage the unique properties of Dirac fermions.
Investigating hybrid systems that combine tunneling control with superconductivity, strong magnetic fields, or moiré engineering presents a particularly promising frontier for future technological developments.

The transition from fundamental scientific curiosity to applied quantum engineering underscores graphene's remarkable adaptability and potential for transformative applications.
Addressing current technical limitations while extending the established concepts to valleytronics and multilayer systems offers concrete research opportunities that can translate the unique physics of Dirac fermions into revolutionary electronic and optoelectronic technologies.

In the next section, the physical-mathematical model adopted in this study will be described in detail.
Building on the conceptual foundation presented earlier regarding quantum dispersion and Rashba-type spin-orbit interaction, the system and corresponding equations that allow numerical simulation of the phenomena observed in our results will be defined.



% Development
    \section{Development}\label{sec:development}
    \subsection{Physical Model}\label{subsec:physical-model}

We model a quantum channel formed by monolayer graphene (MLG) encapsulated on hexagonal boron nitride and capped with SiO$_2$.
Two metallic electrodes generate a perpendicular electric field that guides a Gaussian wave packet (GWP) along the channel.
A rectangular electrostatic barrier defines the scattering region where the SOIR is active.
Related setups and parameters are detailed in our previous article\cite{Serna2019}.

\subsection{Mathematical Model}\label{subsec:mathematical-model}

From the physical model, we can get a mathematical model which describes the temporal evolution and consequently the quantum dispersion of the Dirac fermions in the MLG\@.

Starting with the pristine graphene hamiltonian\cite{Geimk2007}:

\begin{equation}
    \label{eq:pristGr}
    \hat{\bn{H}}_G = v_{\mb{\tiny F}}\vec{\mathbf{\sigma}}\cdot\vec{p},
\end{equation}

\noindent With $\vec{\mathbf{\sigma}} = \hat{\mathbf{\sigma}}_{x}\hat{\imath} + \hat{\mathbf{\sigma}}_{y}\hat{\jmath}$, being the pseudospin Pauli matrices $\hat{\mathbf{\sigma}}_{x} = \bigl(\begin{smallmatrix}
0&1 \\ 1&0
\end{smallmatrix} \bigr)$, $\hat{\sigma}_{y} = \bigl(\begin{smallmatrix}
                                                         0&-i \\ i&0
\end{smallmatrix} \bigr)$, and $\vec{p}=\hat{p}_{x}\hat{\imath}+\hat{p}_{y}\hat{\jmath}$, the momentum operator, which $x, y$ components read $\hat{p}_{x} = -i\hbar\frac{\partial}{\partial x}$ and $\hat{p}_{y} = -i\hbar\frac{\partial}{\partial y}$, respectively.

The standard Rashba coupling for monolayer graphene (often written, in a four-component basis, as $\hat{H}_{\mathrm{DR}}=\lambda_R(\sigma_x s_y-\sigma_y s_x)$, with $s_i$ Pauli matrices in real-spin space) is generally very weak in pristine samples.

In our baseline configuration (pristine regions), we therefore neglect $\hat{H}_{\mathrm{DR}}$, consistent with typical estimates where $\lambda_R \ll \hbar v_F k_0$ and $\lambda_R \ll V_b$ at the scales of interest.

By contrast, in proximity-engineered heterostructures (e.g., graphene on TMDs or with Au), $\lambda_R$ can be enhanced to meV scales\cite{AvsarNatCommun2014, WangPhysRevX2016}.

In such cases, the canonical Dirac--Rashba term must be explicitly included in the Hamiltonian (together with the corresponding four-component spinor structure), and its consequences on wave-packet dynamics and transmission should be analyzed. We delineate this extension at the end of this subsection.


From this point forward, $v_{\mb{\tiny F}}$ stands for the Fermi velocity of the carriers in MLG, which satisfies

\begin{equation}
    \label{eq:vF}
    v_{\mb{\tiny F}} \approx \frac{c}{300}.
\end{equation}
This order-of-magnitude estimate is widely used for graphene\cite{Geimk2007}.

The previous Hamiltonian can be rewritten as:

\begin{equation}
    \label{eq:pristineGrapheneMatrix}
    \hat{\bn{H}}_G = -i\hbar v_{\mb{\tiny F}}
    \begin{pmatrix}
        0                                                         & \frac{\partial}{\partial x}- i\frac{\partial}{\partial y} \\
        \frac{\partial}{\partial x}+ i\frac{\partial}{\partial y} & 0
    \end{pmatrix}.
\end{equation}

Assuming that the momentum-dependent term of the Rashba Hamiltonian for Q1D (quasi-one dimensional) semiconductor hetero-structures can be extended to the context of MLG-Q1D\cite{RDiago2010, Serna2019, RCDiagoEPL2015}, it follows that the Hamiltonian for the SOIR can be written as:

\begin{equation}
    \label{eq:SOIRHamiltonian}
    \hat{\bn{H}}_R =
    \begin{pmatrix}
        0 & k_{\alpha-} + k_{\beta-} \\
        k_{\alpha+} + k_{\beta+} & 0
    \end{pmatrix},
\end{equation}

\noindent Where $k_{\alpha\pm} = \alpha\left(k_{\pm}^2/k_{\mp}\right)$ and $k_{\beta\pm} = \beta k_{\pm}^3$; also defining that $k_{\pm}=k_x\pm i k_y$ which are the initial wave numbers of the system.
The symbols $\alpha$ and $\beta$ represent the linear and cubic contribution of the SOIR respectively.

\textcolor{red}{Following the symmetry-based arguments and k\,$\cdot$\,p analysis summarized in Ref.~\cite{Serna2019} and the derivation for 1D-confined heavy holes in Ref.~\cite{RCDiagoEPL2015}, the leading Rashba invariants under structural inversion asymmetry and quasi-1D confinement produce off-diagonal momentum structures combining a term proportional to $k_{\pm}^2/k_{\mp}$ and a cubic term proportional to $k_{\pm}^3$.
Our model adopts precisely these invariants as an \emph{effective} SOIR inside the barrier region to emulate strong spin--momentum coupling under inversion asymmetry and transverse-mode selection ($k_y\!\to\!0$).
This transfer of functional form is justified at the level of symmetry and low-energy gradient expansion: (i) the barrier region breaks inversion along $z$, (ii) transport is operated near a single transverse mode (Q1D), and (iii) only in-barrier states feel the effective SOIR\@.
Therefore, employing the heavy-hole Rashba Hamiltonian as an effective barrier term in our Q1D graphene setup is consistent with the arguments reviewed in Ref. \cite{Serna2019} and the construction in Ref. \cite{RCDiagoEPL2015}.}

To calculate the values of the coupling constants $\alpha$ and $\beta$, we base our approach on the equations proposed by Wong and Mireles\cite{WongUNAM2005}:

\begin{align}
    \alpha &= \frac{eE_b P^2}{3E_i\left( E_i + E_g \right)}\label{eq: alfa}\\
    \beta &= -\frac{eE_b P^2\left( 2E_i + E_g \right)}{3E_i\left( E_i + E_g \right)^2 k_c^2}\label{eq: beta}\\
    E_i &= \frac{2P^2 k_c^2}{3E_g} \label{eq: E_i}\\
    \frac{2P^2}{m_0 E_g} &= \frac{m_0}{m^*} - 1 \label{eq: P2}\\
    E_b &= \frac{V_b}{el} \label{eq: E_b}
\end{align}

\noindent \textcolor{red}{where $e$ is the elementary charge, $P$ is the momentum matrix element, $E_g$ is the band gap, $k_c$ is a critical wave number that satisfies $k \ll k_c$, $m_0$ is the free-electron mass, $m^*$ is the effective mass used to parametrize the barrier region, $V_b$ is the barrier height, and $l$ is the barrier width.
In our modeling, carriers remain graphene electrons; the heavy-hole Rashba functional form is used only as an effective SOIR inside the barrier.}

Based on previous experimental data\cite{HuntSci2013, FuhrerSci2013, PallaBullMaterSci2016}, we define, as a reference, $E_g = 0.03$ eV and $m^* = 0.47m_0$; we also choose a critical wave number $k_c = 0.2$ \AA$^{-1}$ consistent with reported values.

If we substitute these numerical values, we can use~\eqref{eq: P2} to obtain $P^2=1.54\times10^{-32}$ kg$\cdot$eV; and, therefore, $E_i = 1.37\times10^{-32}$ eV\@.

Finally, we define the width and height of the barrier based on different physical models already presented in the literature.

Using these numerical results, we can find $\alpha$ and $\beta$.
For example, if $l=100$ \AA\, and $V_b = 0.5$ eV, then $\alpha = 0.25$ eV$\cdot$\AA\, and $\beta = -1.56$ eV$\cdot$\AA.

%alpha should be between 0.06 and 0.4 evA

%For monolayer graphene in contact with hexagonal Boron Nitride, approximately 30meV has been calculated

%We are also taking into account that $\alpha = -\beta$, following the conclusions presented by Wong and Mireles\cite{WongUNAM2005}.

\textcolor{red}{For this research, we do not change the carrier type (Dirac fermions); rather, we assume that only the barrier region hosts an effective Rashba-like SOIR with the momentum dependence in Eq.~\eqref{eq:SOIRHamiltonian}, while pristine regions remain Rashba-free.}

The potential barrier is defined only in the specified region.

With the following Hamiltonian:

\begin{equation}
    \label{eq:potentialBarrierHamiltonian}
    \hat{\bn{H}}_V=\bn{I}_2 V(x)
\end{equation}

Adding up the graphene\eqref{eq:pristineGrapheneMatrix}, the SOIR\eqref{eq:SOIRHamiltonian}, and the potential barrier\eqref{eq:potentialBarrierHamiltonian} Hamiltonians, while also zeroing the parameters in the $y$ direction, we get:

\begin{equation}
    \label{eq:hybridHamiltonian}
    \hat{\bn{H}}=
    \begin{pmatrix}
        V(x) & \left(k_{\alpha-}+k_{\beta-}\right)-i\hbar v_{\mb{\tiny F}}\frac{\partial}{\partial x} \\
        \left(k_{\alpha+}+k_{\beta+}\right)-i\hbar v_{\mb{\tiny F}}\frac{\partial}{\partial x} & V(x)
    \end{pmatrix}.
\end{equation}

Note that if we wanted to see the results in the $y$ direction, we would have to zero the partial derivatives of $x$.

If the experimental context implies a giant Rashba coupling (e.g., proximity to TMDs, Au intercalation/adatoms), the Hamiltonian should be augmented to
(i) a four-component spinor (sublattice $\otimes$ real spin) and
(ii) include $\hat{H}_{\mathrm{DR}}=\lambda_R(\sigma_x s_y-\sigma_y s_x)$ along with $\hat{\bn{H}}_G$ and the electrostatic potential.
Within our time-evolution framework, this amounts to propagating a four-component Gaussian wave packet under the enlarged $4\times 4$ operator (the finite-difference machinery and stability criteria carry over directly).
Qualitatively, one should then expect pronounced spin splitting, spin precession inside and near the barrier, spin-dependent resonant features, and the possibility of transitions between Klein and anti-Klein tunneling regimes under suitable parameters\cite{DellAnnaJPhysCondMatt2018, AvsarNatCommun2014, WangPhysRevX2016}.

This is the basis of a more general model that can be explored in future work, particularly in light of recent experimental advances in engineering strong Rashba coupling in graphene-based heterostructures\cite{ManchonNatureMater2015}.

\subsection{Quasi-1D confinement and sub-band hybridization}\label{subsec:q1d-confinement-hybridization}
We model transport along $x$ under quasi-one-dimensional (Q1D) conditions produced by a transverse ($y$) electrostatic confinement that creates discrete transverse modes.
In practice, this can be achieved by electrostatic guiding (smooth $p$-$n$-$p$ profiles or split-gate definition) or narrow channels, leading to a ladder of sub-bands indexed by $n$ in the $y$ direction\cite{Young2009, ParedesPhysRevB2021, TrauzettelNature2007}.
Two limiting parametrizations are useful for estimates:
(i) hard-wall guiding of width $W$, where the transverse quantization gives $k_{y,n}\approx (n+\delta)\pi/W$ and a sub-band spacing $\Delta_y\sim \hbar v_F\pi/W$ between $n=0$ and $n=1$; and
(ii) approximately parabolic electrostatic confinement characterized by $\hbar\omega_y$, yielding a mode spacing $\Delta_y\simeq \hbar\omega_y$.

In our simulations, the barrier is uniform along $y$ (i.e., $V=V(x)$), and we set $\partial_y\to 0$ consistently with a single-mode approximation centered at $k_y\simeq 0$.
This is justified provided the injected Gaussian packet populates predominantly the lowest transverse mode, and inter-subband couplings are negligible.
Quantitatively, we impose the single-mode criteria
\begin{equation*}
\Delta_y \gg \max\{\lambda_R,\ \sigma_E,\ \hbar v_F\,\Delta k_y\},
\end{equation*}
\noindent where $\sigma_E$ is the energy spread of the packet and $\Delta k_y\sim 1/\sigma_y$ is its transverse-$k$ width (set by the real-space transverse width $\sigma_y$ of the injection).
For hard-wall guiding, this amounts to $\hbar v_F\pi/W \gg \{\lambda_R,\sigma_E,\hbar v_F/\sigma_y\}$; for smooth confinement, to $\hbar\omega_y \gg \{\lambda_R,\sigma_E,\hbar v_F/\sigma_y\}$.
Under these conditions, sub-band hybridization is suppressed and the Q1D reduction with $k_y\to 0$ (and $y$-derivatives zeroed) is controlled.

Sub-band hybridization can become important if (a) the confinement is weak (small $\Delta_y$ or wide $W$), (b) the wave packet significantly occupies multiple $k_y$ components, (c) the potential varies along $y$ (i.e., $V=V(x,y)$) producing mode-mixing matrix elements $\propto\langle n|\partial_y V|m\rangle$, or (d) spin-orbit terms couple modes with different transverse parity.
In such regimes, one must retain the full 2D problem or expand the spinor in transverse eigenmodes,
$\Psi(x,y,t)=\sum_n \chi_n(y)\, \varphi_n(x,t)$,
leading to coupled 1D channel equations with inter-mode couplings determined by $V(x,y)$ and Rashba matrix elements.
Hybridization then yields avoided crossings and multi-channel resonances in transmission (Fano-like features), and the effective single-channel picture no longer applies\cite{MiroshnichenkoRevModPhys2010}.

We have verified that the parameter sets used in this work satisfy the single-mode criteria above, ensuring that (i) only the lowest transverse mode is appreciably occupied at injection and (ii) inter-subband couplings driven by the barrier and Rashba terms remain negligible.
Consequently, the confinement term is justified and sub-band hybridization does not alter our conclusions within the explored parameter window.
Exploring multi-mode transport and intentional mode hybridization (e.g., by patterned $V(x,y)$ or stronger $\lambda_R$) is a natural extension of the present framework.

Note that if we wanted to see the results in the $y$ direction, we would have to zero the partial derivatives of $x$.

\subsection{Finite differences scheme}\label{subsec:finite-differences-scheme}

The main equation to solve is the time-dependent eigenvalue equation:

\begin{equation}
    \label{eq:schrodingerTimeDependent}
    i\hbar \frac{d}{dt}\bn{\Psi}(x,t) = \hat{\bn{H}}\bn{\Psi}(x,t)
\end{equation}

In our case, we consider a Gaussian wave packet (GWP) for each component of the pseudospinor $\bn{\Psi}(x,t)=\begin{psmallmatrix*}
\psi_A(x,t)\\\psi_B(x,t)
\end{psmallmatrix*}$.
The GWP at an initial time has the following form:

\begin{equation}
    \label{eq:GWP}
    \psi_j(x,t_0)=\frac{\xi_j}{\sqrt[4]{\pi\sigma^2}}e^{-\frac{(x-x_0)^2}{2\sigma^2}}e^{ixk_0}
\end{equation}

\noindent Where $\xi$ is the initial configuration of the pseudospinor, $k_0$ is the initial wave number in $x$, and the subscript $j$ indicates the component being used.

For the development of this article, we are using the following pseudospinor configuration options:

\begin{equation}
    \label{eq:pseudospinorConfigurations}
    \xi=\begin{pmatrix}
            1\\0
    \end{pmatrix},\,\,\begin{pmatrix}
                          1\\1
    \end{pmatrix},\,\,\begin{pmatrix}
                          1\\i
    \end{pmatrix},\,\,\begin{pmatrix}
                          1\\e^{i\pi/4}
    \end{pmatrix}.
\end{equation}

The wave packet is then defined as follows:

\begin{equation}
    \label{eq:FinalGWP}
    \bn{\Psi}(x,0)=\frac{1}{\sqrt {\xi_A^2 + \xi_B^2}}\begin{pmatrix}
                                                          \psi_A\\\psi_B
    \end{pmatrix}
\end{equation}

For the probability density calculated below, we use the following definition:

\begin{equation}
    \label{eq:probDens}
    \rho(x,t)=|\bn{\Psi}|^2 = |\psi_A|^2+|\psi_B|^2
\end{equation}

\subsection{Temporal Evolution Operator}\label{subsec:temporal-evolution-operator}
To solve the eigenvalue equation\eqref{eq:schrodingerTimeDependent}, we propose using a temporal evolution operator (TEO), based on the following equation:

\begin{equation}
    \label{eq:TEOdifEq}
    \bn{\Psi}(x,t)=\hat{U}(t,t_0)\bn{\Psi}(x,t_0)
\end{equation}

\noindent We can see that by applying the operator, we obtain a time $t$ from an initial time $t_0$.
If we substitute the TEO equation\eqref{eq:TEOdifEq} into the eigenvalue equation\eqref{eq:schrodingerTimeDependent}, we can develop the algebra and solve the differential equation to obtain the TEO\@.
Since the Hamiltonian does not depend on time, we can rewrite eq.\eqref{eq:TEOdifEq} as:

\begin{equation}
    \label{eq:TEOApplied}
    \bn{\Psi}(x,t)= e^{-\frac{i\delta t}{\hbar}\hat{\bn{H}}} \bn{\Psi}(x,t_0)
\end{equation}

\noindent Where $\delta t$ is the time step that we will discretize later.
The previous exponential can be approximated through the first-order Taylor series as:

\begin{equation}
    \label{eq:cayleyApprox}
    e^{-\frac{i\delta t}{\hbar}\hat{\bn{H}}} = \frac{2}{\bn{I} + \frac{i\delta t}{2\hbar}\hat{\bn{H}}}-\bn{I}
\end{equation}

Using a change of variable, we can rewrite the previous equation\eqref{eq:cayleyApprox} as:

\begin{equation}
    \label{eq:systemOfEquations}
    \bn{\Phi}(x,t_0) + \frac{i\delta t}{2\hbar}\hat{\bn{H}}\bn{\Phi}(x,t_0)=2\bn{\Psi}(x,t_0)
\end{equation}

\noindent This equation represents a system of $2\times2$ equations (since we have two unknowns, $\phi_A$ and $\phi_B$, and when developing the matrix, we can see that two equations are formed).

Since both equations have derivatives, we use the finite difference method to solve both equations for the entire space simultaneously.
To do this, we first discretize the system of equations\eqref{eq:systemOfEquations} along the entire space $j$ from $0$ to $J$.

To maintain the stability of the numerical analysis in the finite difference method, the following condition must be respected\cite{Carrillo2015}:

\begin{equation}
    \label{eq:stabilityCondition}
    \delta t \leq \frac{\left( \delta x \right)^2}{2}
\end{equation}

To discretize the derivatives, we use the Taylor series expansion up to the first degree for the point $x_j$ forward and backward:

\begin{align}
    \label{eq:TaylorBeforeAndAfter}
    f(x_j+\delta x)&=f(x_j) + (x_j+\delta x-x_j)f'(x_j)\nonumber\\
    &=f(x_j)+\delta xf'(x_j),\nonumber\\
    f(x_j-\delta x)&=f(x_j) + (x_j-\delta x-x_j)f'(x_j)\nonumber\\
    &=f(x_j)-\delta xf'(x_j),
\end{align}

\noindent if we subtract the second equation from the first and solve for $f'(x_j)$, we can find the ``Central Differences'' which we can also discretize as:

\begin{equation}
    \label{eq:diferenciasCentradas}
    f'(x_j)=\frac{f(x_j+\delta x)-f(x_j-\delta x)}{2\delta x} = \frac{f_{j+1}-f_{j-1}}{2\delta x}
\end{equation}

In this way, we can rewrite our system of equations, once the discretization and algebraic development of the matrix have been performed:

\begin{equation}
        \label{eq:partA}
    \phi_{A, j} + \frac{i\delta t}{2\hbar}(V_j\phi_{A,j} + \left(k_{\alpha-,j}+k_{\beta-,j}\right)\phi_{B,j}-
    i\hbar v_{\mb{\tiny F}}\frac{\phi_{B, j+1}-\phi_{B, j-1}}{2\delta x}) = 2\psi_{A,j}
\end{equation}

\begin{equation}
    \label{eq:partB}
    \phi_{B, j} + \frac{i\delta t}{2\hbar}(V_j\phi_{B,j} + \left(k_{\alpha+,j}+k_{\beta+,j}\right)\phi_{A,j}-
    i\hbar v_{\mb{\tiny F}}\frac{\phi_{A, j+1}-\phi_{A, j-1}}{2\delta x}) = 2\psi_{B,j}
\end{equation}

If we add Eq.\eqref{eq:partA} and Eq.\eqref{eq:partB}:


\begin{align}
    \label{eq:temporaryEquation}
    2\begin{pmatrix}
         \psi_{A,1} + \psi_{B,1} \\
         \psi_{A,2} + \psi_{B,2} \\
         \vdots                  \\
         \psi_{A,J} + \psi_{B,J} \\
    \end{pmatrix} &=
    \begin{pmatrix}
        N      & Q      & 0      & \ldots & \ldots & \ldots & 0 \\
        -Q     & N      & Q      & 0      & \ldots & \ldots & 0 \\
        0      & -Q     & N      & Q      & 0      & \ldots & 0 \\
        \vdots & 0      & -Q     & N      & Q      & \ddots & 0 \\
        \vdots & \vdots & 0      & -Q     & N      & \ddots & 0 \\
        \vdots & \vdots & \vdots & \ddots & \ddots & \ddots & Q \\
        0      & 0      & 0      & 0      & 0      & -Q     & N
    \end{pmatrix}\begin{pmatrix}
                     \phi_{A,1} \\
                     \phi_{A,2} \\
                     \vdots     \\
                     \phi_{A,J}
    \end{pmatrix}\\
    &+\begin{pmatrix}
          M      & Q      & 0      & \ldots & \ldots & \ldots & 0 \\
          -Q     & M      & Q      & 0      & \ldots & \ldots & 0 \\
          0      & -Q     & M      & Q      & 0      & \ldots & 0 \\
          \vdots & 0      & -Q     & M      & Q      & \ddots & 0 \\
          \vdots & \vdots & 0      & -Q     & M      & \ddots & 0 \\
          \vdots & \vdots & \vdots & \ddots & \ddots & \ddots & Q \\
          0      & 0      & 0      & 0      & 0      & -Q     & M
    \end{pmatrix}\begin{pmatrix}
                     \phi_{B,1} \\
                     \phi_{B,2} \\
                     \vdots     \\
                     \phi_{B,J}
    \end{pmatrix}
\end{align}


\noindent where $N =1 + \frac{i\delta t}{2\hbar} \left( V_j + \left(k_{\alpha+,j}+k_{\beta+,j}\right)\right)$, $Q=\frac{i\hbar v_{\mb{\tiny F}}}{2\delta x}$ and $M =1 + \frac{i\delta t}{2\hbar}\left( V_j + \left(k_{\alpha-,j}+k_{\beta-,j}\right)\right)$.

As can be seen, the size of the matrix depends on the size of our system.
If we choose $\delta x = 1\angstrom$, then a quantum well of $1200\angstrom$ will generate a matrix of $1200\times1200$.

We repeat the previous procedure but now subtracting Eq.\eqref{eq:partA} and Eq.\eqref{eq:partB}

Looking at these matrices, we can redefine them as two simple matrix equations $2\psi_+=\mathbb{N}\phi_A+\mathbb{M}\phi_B$ and $2\psi_-=\mathbb{L}\phi_A+\mathbb{P}\phi_B$.
These equations can be treated as a system of matrix equations that are solved as follows:

\begin{align}
    \label{eq:sistemaMatricial}
    \phi_A&=(2\mathbb{N}^{-1})\psi_+-(\mathbb{N}^{-1}\mathbb{M})\phi_B\nonumber\\
    \phi_B&=2(-\mathbb{L}\mathbb{N}^{-1}\mathbb{M}+\mathbb{P})^{-1}(\psi_--\mathbb{L}\mathbb{N}^{-1}\psi_+)
\end{align}

Once we have the discretized system of equations, the temporal evolution can be calculated with:

\begin{equation}
    \label{eq:siguienteTiempo}
    \Psi_j^{n+1}=\Phi_j^n-\Psi_j^n
\end{equation}

Repeating the process for each $n$ until reaching $N$.

\subsection{Probability Current Density}\label{subsec:probability-current-density}

Once we have the temporal evolution, we can calculate its transmission coefficient from the calculation of the probability current density (PCD).
To find this current, we start from the following continuity equation:

\begin{equation}
    \label{eq:continuidad}
    \frac{\partial\rho}{\partial t} + \nabla\cdot\vec{j}=0
\end{equation}

Applying the conjugate transpose to the time-dependent eigenvalue equation\eqref{eq:schrodingerTimeDependent}, and multiplying on the right by $\Psi$, we can incorporate the probability density equation\eqref{eq:probDens}.
At the same time, we consider the graphene Hamiltonian\eqref{eq:pristGr}, which we know is Hermitian since it is composed of Pauli matrices.
In this way, we can construct the following equation:

\begin{equation}
    \label{eq:rho_sigma_nabla}
    \frac{\partial}{\partial t}\int\rho d\vec{r} = -v_f\int\left( \Psi\vec{\sigma}\cdot\nabla\Psi^{T*} + \Psi^{T*}\vec{\sigma}\cdot\nabla\Psi \right)d\vec{r}
\end{equation}

We can rewrite the equation by factoring out $\nabla$ and removing the integral from both sides:

\begin{equation}
    \label{eq:casi_j}
    \frac{\partial}{\partial t}\rho = -v_f\nabla\cdot\left( \Psi^\dagger \vec{\sigma} \Psi \right)
\end{equation}

Substituting into the continuity equation of probability density eq.\eqref{eq:continuidad}:

\begin{equation}
    \label{eq:nablaJ}
    \nabla\cdot\vec{j} = v_f\nabla\cdot\left( \Psi^\dagger \vec{\sigma} \Psi \right)
\end{equation}

Eliminating $\nabla$ from both sides:

\begin{equation}
    \label{eq:final}
    \vec{j} = v_f\Psi^\dagger \vec{\sigma} \Psi
\end{equation}

In this way, we have obtained the probability current density in graphene.

If we want to obtain the components of the current density, we can do it as follows:

\begin{equation}
    \label{eq:componentes}
    \vec{j} = v_f\begin{pmatrix}
                     \Psi^\dagger \sigma_x \Psi \\ \Psi^\dagger \sigma_y \Psi
    \end{pmatrix} = v_f\begin{pmatrix}
                           \Psi_A^\dagger\Psi_B + \Psi_B^\dagger\Psi_A \\ i(-\Psi_A^\dagger\Psi_B + \Psi_B^\dagger\Psi_A)
    \end{pmatrix}
\end{equation}

The importance of the current vector expression, shown explicitly in equation\eqref{eq:componentes}, lies in the detailed understanding of the pseudospin behavior of graphene.
The clear characterization of both components of the current vector is central to interpreting the numerical results and understanding the phenomenology associated with transport in these structures.

As a verification, if we consider an incident plane wave function, we can express it as:

\begin{equation}
    \label{eq:ondaPlana}
    \Psi(\vec{r})=\frac{e^{i\vec{k}\cdot\vec{r}}}{\sqrt{2}}
    \begin{pmatrix}
        1 \\ e^{i\theta}
    \end{pmatrix},
\end{equation}

\noindent where $\theta$ is the angle of incidence with respect to the direction normal to the barrier and $\vec{k} = k(\cos\theta,\sin\theta)$ is the wave vector.

If we substitute eq.\eqref{eq:ondaPlana} into eq.\eqref{eq:componentes}, we can simply perform the multiplication as follows to obtain the probability current density incident on the barrier along the $X$ axis ($j_{x,in}$):

\begin{align}
    \label{eq:jdemostrada}
    j_{x, in}&=v_f\frac{1}{2}
    \begin{pmatrix}
        e^{-i\theta} & 1
    \end{pmatrix}
    \begin{pmatrix}
        0 & 1 \\
        1 & 0
    \end{pmatrix}
    \begin{pmatrix}
        1 \\ e^{i\theta}
    \end{pmatrix}\nonumber \\
    &=v_f\left( \frac{e^{-i\theta} + e^{i\theta}}{2} \right)\nonumber \\
    &=v_f\cos\theta
\end{align}

This simplification is what is commonly used in the literature\cite{DahalJPhysChemSolids2017, WuJAP2009}.

To determine the numerical value of $\vec{\jmath}$, we identify the peaks in the probability density plots.
Subsequently, we calculate the minima of the Gaussian curves to find the apparent width of the GWP. Finally, we locate two precise moments: immediately before the wave packet interacts with the potential barrier and the exact moment it has traversed the barrier.

With the defined entry ($t_1$) and exit ($t_2$) times, we use Eq.\eqref{eq:componentes} for the entire space at each of these times.

With this, we can now calculate the transmission coefficient:

\begin{equation}
    \label{eq:transmissionCoef}
    T = \left| \frac{\vec{\jmath}_{out}\cdot\hat{n}}{\vec{\jmath}_{in}\cdot\hat{n}} \right|,
\end{equation}

\noindent In this context, the ratio between the transmitted current and the incident current is calculated, considering the product with the vector normal to the barrier.
In the one-dimensional case addressed here, this vector can be represented as $(1,0)$ or $(0,1)$, depending on whether the Gaussian wave packet (GWP) propagates along the $x$-axis or the $y$-axis, respectively.

Having presented the conceptual framework and the fundamental equations of the studied system, we now proceed with the analysis and critical discussion of the results obtained through numerical simulations.
This section will allow us to evaluate and contrast in detail the influence of the Rashba term on quantum transmission in monolayer graphene.


    \section{Discussion of Results}\label{sec:discussion-of-results}
        A continuación se muestran las figuras obtenidas con el procedimiento obtenido anteriormente.

    \begin{figure}[h!]
        \centering
        \begin{minipage}[t]{0.48\textwidth}
            \centering
            \includegraphics[width=\textwidth]{../assets/images/No-Rashba/TCoefficient(1.0,0)xalpha=0beta=0}
            \caption{figure}{
Transmission coefficient (T) in pristine graphene with initial pseudospinor configuration $\xi = (1, 0)$, plotted against potential barrier height (Vb, in meV) and initial wave vector ($k_0$, in $\angstrom^{-1}$). The 3D plot and 2D heatmap show that transmission is largely independent of the initial wave vector but decreases noticeably as the barrier height increases, ranging from $1$ to approximately $0.990$.
}
            \label{fig:noRashba}
        \end{minipage}
        \hfill
        \begin{minipage}[t]{0.48\textwidth}
            \centering
            \includegraphics[width=\textwidth]{../assets/images/Rashba/TCoefficient(1.0,0)xalpha=0.2beta=-0.2}
            \caption{figure}{
Coeficiente de transmisión ($T$) en función de la altura de la barrera de potencial ($Vb$) y el número de onda inicial ($k_0$) con una configuración inicial de pseudoespinor $\xi = (1, 0)$. La superficie 3D y el mapa de color 2D muestran una dependencia no monotónica de $T$ con respecto a $Vb$ y $k_0$, destacando la influencia del acoplamiento espín-órbita en la transmisión a través de la barrera.
}
            \label{fig:rashba}
        \end{minipage}
    \end{figure}

    En las imágenes podemos notar ciertos aspectos interesantes, empezando con la fig.\ref{fig:noRashba}.
    La imagen presenta dos gráficos que ilustran el coeficiente de transmisión ($T$) en función de $Vb$ (en meV) y $k_0$ (en $\angstrom^{-1}$) para un valor fijo de $\xi = (1, 0)$.
    El gráfico de la izquierda es una visualización tridimensional donde $T$ está en el eje $z$, $Vb$ en el eje $x$ y $k_0$ en el eje $y$, con una escala de colores que varía desde morado oscuro (menor $T$) hasta amarillo brillante (mayor $T$). El gráfico de la derecha es un mapa de calor bidimensional que ofrece una vista superior, con el eje $x$ representando $k_0$ y el eje $y$ representando $Vb$, usando el mismo mapa de colores que el gráfico 3D\@.
    Ambos gráficos muestran que el coeficiente de transmisión generalmente se mantiene muy cerca de 1, indicando una alta transmisión.
    Conforme aumenta $Vb$, $T$ tiende a disminuir, mientras que la dependencia respecto a $k_0$ es mínima.
    En general, los gráficos demuestran cómo cambia $T$ con variaciones en $Vb$ y $k_0$, destacando una leve disminución de $T$ al aumentar $Vb$ y un cambio insignificante respecto a $k_0$.

    Este resultado muestra una contradicción con lo que ya se especifíca en la literatura, en el grafeno se esperaría observar una transmisión total debido a las mismas propiedades del material\cite{horsell2008, Young2009}.

    The previously described behavior might be the result of one or both of the following aspects:

    \begin{itemize}
        \item Wave Packet Characteristics:
        Not a single momentum eigenstate.
        Debido a que estamos ocupando paquetes de onda en la simulación, un detalle importante que surge es que existe la presencia de múltiples componentes de momentum y eso implica que algunos de ellos pueden causar interferencia con otros; causando un coeficiente de transmisión que difiere del de un solo autoestado\cite{Staelens2021}.
        Además, en nuestra simulación se está tomando en cuenta el tiempo.
        Parte de la investigación a futuro es comprobar si el tiempo de fase o de tunelaje están directamente relacionados a la variación del coeficiente de transmisión observada en este trabajo.
        \item Quantum interference effects\cite{MolgadoMex2018}.
        Dependiendo del ancho del paquete y su amplitud, este se puede dispersar con el tiempo y esto puede causar interferencia consigo mismo.
        Conforme diferentes partes del GWP encuentran la barrera de potencial, estos pueden estar interfiriendo, causando fluctuaciones en el coeficiente de transmisión.
    \end{itemize}

    Por otro lado, tenemos la transmisión bajo presencia de SOIR (Fig.\ref{fig:rashba}).

Se observa una correlación entre las variables: generalmente, a mayor altura de la barrera de potencial ($Vb$) y mayor número de onda inicial ($k_0$), se espera una menor transmisión.
Sin embargo, la interacción espín-órbita introduce un comportamiento más complejo.
A medida que $k_0$ aumenta y $Vb$ disminuye, la transmisión se acerca a 1, indicando una mayor probabilidad de tunelamiento cuántico debido a la SOIR\@.

Aunque la variación en el coeficiente de transmisión es pequeña (del orden de $10^{-2}$), es significativa y atribuible a la interacción SOIR. Los electrones, con sus diferentes componentes de pseudoespín, interactúan, y esta interacción se ve afectada por el número de onda inicial del paquete de ondas gaussiano (GWP), como se observó en estudios previos\cite{Serna2019}.
Por lo tanto, la SOIR modula la interacción, lo que a su vez causa las fluctuaciones observadas en el coeficiente de transmisión.





% Conclusions

    \section{Conclusions}\label{sec:conclusions}

    In this study, we investigated theoretically and computationally the influence of Rashba spin-orbit interaction (SOIR) on electron transmission across a potential barrier in monolayer graphene.
    We demonstrated that introducing SOIR significantly modifies electron transmission coefficients, generating complex patterns that are absent in pristine graphene.

    An interesting contradiction emerged regarding electron transmission observed in pristine graphene compared to existing literature.
    While pristine graphene commonly exhibits near-perfect transmission due to suppressed electron backscattering, our results indicated small yet notable transmission reductions.
    These deviations were attributed to characteristics of the Gaussian wave packet used, specifically quantum interference phenomena, highlighting the sensitivity of electron transport properties to subtle variations in experimental and simulation conditions.

    Moreover, the inclusion of rashba SOIR introduced additional small but meaningful variations in transmission coefficients.
    This effect was closely linked to the interaction between electron pseudo-spin components, influenced distinctly by the direction and magnitude of the initial wave vector.
    Our observations underscore the delicate yet impactful role pseudo-spin dynamics play in modifying electron transport properties within graphene-based systems.

    Future studies could focus on systematically analyzing the relationship between phase time or tunneling times and these observed transmission variations induced by SOIR. Such exploration could offer deeper insights into the quantum mechanical principles governing transport phenomena in graphene and related materials.

    Ultimately, our findings highlight the potential of exploiting Rashba spin-orbit interaction to precisely control electron transmission through potential barriers, paving the way toward innovative electronic and spintronic devices, as well as advancing quantum computing technologies.


%% Acknowledgments
%    \section*{Acknowledgments}

% References
    \bibliographystyle{apsrev4-2}
    \bibliography{../bib/main}

\end{document}