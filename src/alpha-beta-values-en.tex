To calculate the values of the coupling constants $\alpha$ and $\beta$, we base our approach on the equations proposed by Wong and Mireles\cite{WongUNAM2005}:

\begin{align}
    \alpha &= \frac{eE_b P^2}{3E_i\left( E_i + E_g \right)}\label{eq: alfa}\\
    \beta &= -\frac{eE_b P^2\left( 2E_i + E_g \right)}{3E_i\left( E_i + E_g \right)^2 k_c^2}\label{eq: beta}\\
    E_i &= \frac{2P^2 k_c^2}{3E_g} \label{eq: E_i}\\
    \frac{2P^2}{m_0 E_g} &= \frac{m_0}{m^*} - 1 \label{eq: P2}\\
    E_b &= \frac{V_b}{el} \label{eq: E_b}
\end{align}

\noindent where $e$ is the charge of heavy holes in the potential barrier region, $P$ is the momentum of the charge carriers, $E_g$ is the energy of the band gap, $k_c$ is a critical wave number that satisfies the condition $k \ll k_c$, $m_0$ is the free electron mass, $m^*$ is the effective mass of charge carriers in the potential barrier region, $V_b$ is the height of the barrier, and $l$ is the width of the barrier.

Based on previous experimental data\cite{HuntSci2013, FuhrerSci2013, PallaBullMaterSci2016}, we define, as a reference, $E_g = 0.03$ eV and $m^* = 0.47m_0$; we also choose a critical wave number $k_c = 0.2$ \AA$^{-1}$ consistent with reported values.

If we substitute these numerical values, we can use~\eqref{eq: P2} to obtain $P^2=1.54\times10^{-32}$ kg$\cdot$eV; and, therefore, $E_i = 1.37\times10^{-32}$ eV\@.

Finally, we define the width and height of the barrier based on different physical models already presented in the literature.

Using these numerical results, we can find $\alpha$ and $\beta$.
For example, if $l=100$ \AA\, and $V_b = 0.5$ eV, then $\alpha = 0.25$ eV$\cdot$\AA\, and $\beta = -1.56$ eV$\cdot$\AA.

%alpha should be between 0.06 and 0.4 evA

%For monolayer graphene in contact with hexagonal Boron Nitride, approximately 30meV has been calculated