Para poder calcular los valores de las constantes de acoplamiento $\alpha$ y $\beta$, nos basamos en las ecuaciones propuestas por Wong y Mireles\cite{WongUNAM2005}:

\begin{align}
    \alpha &= \frac{eE_b P^2}{3E_i\left( E_i + E_g \right)}\label{eq: alfa}\\
    \beta &= -\frac{eE_b P^2\left( 2E_i + E_g \right)}{3E_i\left( E_i + E_g \right)^2 k_c^2}\label{eq: beta}\\
    E_i &= \frac{2P^2 k_c^2}{3E_g} \label{eq: E_i}\\
    \frac{2P^2}{m_0 E_g} &= \frac{m_0}{m^*} - 1 \label{eq: P2}\\
    E_b &= \frac{V_b}{el} \label{eq: E_b}
\end{align}

\noindent donde $e$ es la carga de los huecos pesados en la zona de la barrera de potencial, $P$ es el momentum de los portadores de carga, $E_g$ es la energía del ``band gap'', $k_c$ es un número de onda crítico que cumple la condición $k \ll k_c$, $m_0$ es la masa del electrón libre, $m^*$ la masa efectiva de los portadores de carga en la zona de la barrera de potencial, $V_b$ es la altura de la barrera y $l$ es el ancho de la barrera.

Tomando como base datos experimentales previos (\cite{HuntSci2013, FuhrerSci2013, PallaBullMaterSci2016}) definimos a manera de prueba $E_g = 0.03$ eV y $m^* = 0.47m_0$; mientras que definimos un valor para el número de onda crítico $k_c = 0.2$ \AA$^{-1}$ que coincide con los valores obtenidos experimentalmente.

Si sustituimos estos valores numéricos podemos usar~\eqref{eq: P2} para obtener $P^2=1.54\times10^{-32}$ kg$\cdot$eV; y, por lo tanto, $E_i = 1.37\times10^{-32}$ eV.

Finalmente, el ancho y la altura de la barrera los definimos tomando como base diferentes modelos físicos ya presentados en la literatura.

Usando estos resultados numéricos podemos encontrar $\alpha$ y $\beta$.
Por ejemplo, si $l=100$ \AA\, y $V_b = 0.5$ eV, entonces $\alpha = 0.25$ eV$\cdot$\AA\, y $\beta = -1.56$ eV$\cdot$\AA.

%alfa debe ir entre 0.06 y 0.4 evA

%Para el grafeno monocapa en contacto con nitruro de Boro Hexagonal aproximadamente se ha calculado 30meV
