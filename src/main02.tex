%! Authors = Eduardo Serna & Leo Diago-Cisneros

% Preamble
\documentclass{../assets/templates/iopjournal}
%\documentclass[twocolumn]{revtex4-2}

% Packages
\usepackage{amsmath}
\usepackage{amsfonts}
\usepackage{mathtools}
\usepackage{amsmath,amssymb,graphicx,hyperref}
\usepackage[T1]{fontenc}
\usepackage[utf8]{inputenc}
\usepackage{lmodern}
\usepackage{microtype}
\usepackage{amssymb}
\usepackage{orcidlink}

\newcommand{\bn}[1]{\mbox{\boldmath $#1$}}
\newcommand{\mb}{\mbox}
\newcommand{\angstrom}{\textup{\AA}}

% Document
\begin{document}
%\articletype{Paper}

    \title{Channel-Interference and Spin-Orbit effects on Quantum Tunneling in Mono-layer Graphene}
    
    \author{Eduardo Serna$^1$\orcidlink{0000-0003-2090-7067}, L. Diago-Cisneros$^2$\orcidlink{0000-0001-9409-1545} and I. Rodr\'iguez-Vargas$^3$.}
    
    \affil{$^1$Centro de Investigación en Ciencias, Universidad Autónoma del Estado de Morelos, Av. Universidad 1001 Col. Chamilpa, 62209, Cuernavaca, Morelos, México; sernaed95@gmail.com}
       
    \affil{$^2$Facultad de Física, Universidad de La Habana, La Habana, 10400, Cuba}
    
    \affil{$^3$Unidad Académica de Ciencia y Tecnología de la Luz y la Materia, Universidad Autónoma de Zacatecas, Circuito Marie Curie S/N, Parque de Ciencia y Tecnología QUANTUM Ciudad del Conocimiento, 98160 Zacatecas, Zacatecas, México}
        
    \affil{$^*$L. Diago-Cisneros.Author to whom any correspondence should be addressed}
    \email{ldiago@fisica.uh.cu}
    
    \keywords{graphene, Rashba spin--orbit interaction, wave packets, electron transmission, spintronics}

% Abstract
    \begin{abstract}
        This theoretical study addresses the impact of channel interference (CI) and the Rashba spin-orbit interaction (SOIR) effects, on the transmission of Dirac-fermions (fD) throughout a potential barrier in monolayer graphene (MLG). Our findings reveal that the presence of SOIR significantly modifies fD transmission coefficients compared with those under pristine graphene conditions. Indeed, while quantum scattering in pristine graphene remains near unity, the introduction of SOIR leads to an intricate transmission patterns that depend mainly on the barrier parameters. The quantum transport phenomena are further complicated by channel interference between pseudospinor components, which manifest themselves as cross-coupling terms in the probability current density and significantly influence the scattering behavior of the MLG. These effects arise from the spin degeneracy lifting and spin-dependent scattering mechanisms. We hope that our outcomes supported by the suggested layered device with a gate-voltage embedded MLG, will emphasize enough the critical role of both the SOIR and the CI in forthcoming engineering of potential Spintronics as well as Quantum Computing devices.
    \end{abstract}


% Table of Contents (optional)
%    \tableofcontents
%    \newpage

% Introduction
    \section{Introductory Overview}\label{sec:introduction}
    % Introduction: Quantum Dispersion and Dirac Fermion Tunneling in Graphene

Graphene exhibits a linear dispersion near the Dirac points, leading to massless Dirac-fermion (fD) behavior.
The phenomenon of Klein (perfect tunneling) of fD carriers at normal incidence through high potential barriers, is a hallmark of such systems. Conversely, bilayer graphene (BLG) shows anti-Klein tunneling, (perfect reflection) at normal incidence in the absence of a bandgap. Over the past decade, both theoretical predictions and experimental validations of these phenomena have matured significantly, particularly in monolayer (MLG) and few-layer graphene (FLG) systems.

In pristine MLG, the canonical Dirac–Rashba (DR) term is typically very small (often quoted in the $\mu$eV range), and thus it is commonly neglected for transport at the energy and length scales considered here. This is the rationale behind the absence of the standard DR term in the pristine regions of our model. However, the literature also reports proximity-induced scenarios where a giant Rashba coupling can emerge. For example, in graphene interfaced with transition-metal dichalcogenides or in systems with Au adatoms/intercalation, leading to effective spin-orbit coupling (SOC) values in the meV–tens of meV range\cite{AvsarNatCommun2014, WangPhysRevX2016}.

In such cases, the canonical DR term should be included and its impact on wave-packet propagation analyzed (\textit{e.g.}, stronger spin splitting, spin precession, modified transmission resonances, and even transitions from Klein to anti-Klein regimes). In the present study we focus on the pristine-graphene limit where the canonical term is negligible, while we model proximity-induced effects within the barrier \textit{via} an effective Rashba-like spin orbit (SOIR) coupling, as detailed in Sec.~\ref{subsec:mathematical-model}.

Finally, it is worth noting that one of the essential aspects addressed in this work lies in the analytical formulation of the probability-density current vector, whose mathematical expression explicitly considers the channel interference (CI) -which departs from common approaches-, and is given by \eqref{eq:componentes}. This expression derives from the standard well-established representation of the charge-spin probability-density current operator associated with the DR model in graphene and has been discussed in prior literature\cite{AvishaiPhysRevB2021}. Our main target is to explore the fD tunneling trough a quantum barrier allocated electrostatically in a MLG nanoribbon [see Fig.~\ref{fig:physical-model}]. Thus, we perform a component-wise and correlation-based theoretical analysis tailored by a time-resolved wave-packet under SOIR and CI.  As shown later in Sec.~\ref{sec:discussion-of-results}, the formula \eqref{eq:componentes} provides practical interpretability of the quantum scattering features within the scenario descibed above.

\subsection{Klein Paradox}\label{subsec:kleinParadox}

An electrostatic potential applied to graphene, creates regions where fD are classically forbidden due to energy constraints\cite{SoninPhysRevB2009}. For instance, a gate voltage can shift the local band structure (\textit{i.e}., the Dirac point) relative to the Fermi level, creating a p-type region in which, at the incident fD energy, only valence-band states are available; classically, electrons lack sufficient energy to surmount this potential barrier\cite{DellAnnaJPhysCondMatt2018}.
However, in graphene, fD transmission through such barriers can exhibit unusual behavior due to the Klein paradox\cite{Young2009}.

The Klein paradox describes the phenomenon where fD in graphene can tunnel through electrostatic potential barriers with a probability approaching unity, even at perpendicular incidence, seemingly defying the principles of classical mechanics\cite{TrauzettelNature2007}. This counterintuitive transmission arises from the relativistic nature of fD in graphene, \textit{i.e.} within the barrier region, incident electrons can be converted into holes, allowing for unimpeded passage\cite{BernardiniJPhysAMathTheor2010}. The Klein paradox is a direct consequence of graphene's unique linear dispersion relation and the associated massless fD behavior.
While this perfect transmission through barriers might pose challenges for achieving electron confinement in some device designs, it also presents opportunities for developing novel tunneling-based electronic devices. Beyond simply creating a forbidden region, the application of electrostatic potential can dramatically alter electron transmission through various other mechanisms, influencing the refractive index and leading to phenomena like electron lensing\cite{ParedesPhysRevB2021}.

\subsection{Rashba-Type Spin-Orbit Interaction}\label{subsec:rashba-type-spin-orbit-interaction}

SOIR is a relativistic effect that arises in systems lacking structural inversion symmetry\cite{AvishaiPhysRevB2021}. It becomes especially relevant due to its ability to significantly modify the electronic structure of graphene, inducing particular couplings between the spin and momentum of spin-charge carriers.
This leads to emerging phenomena such as spin-dependent quantum interference and unconventional scattering effects.
The impact of SOIR not only provides new perspectives in Fundamental Physics, but also opens promising possibilities for advanced applications in Spintronics technologies and quantum information-based devices.

In graphene, SOIR can be induced by an external electric field applied perpendicular to the graphene plane or through proximity to a substrate\cite{ShcherbakovSciAdv2021}.
A key consequence of this interplay is the breaking of spin degeneracy, where the energy levels of electrons with opposite spins are no longer the same, even in the absence of an external magnetic field\cite{DelkhoshPhysE2015}.
Controlling spin degeneracy through electric fields via SOIR, is particularly valuable for Spintronics applications.
It enables spin manipulation without depending on magnetic fields, which typically require more power and are challenging to incorporate effectively into nanoscale systems.

Furthermore, SOIR can strongly reshape the band dispersion and spin textures in graphene and related heterostructures\cite{WangPhysRevX2016}.
In proximity-engineered systems that additionally break sublattice symmetry or combine multiple spin–orbit coupling (SOC) channels, gaps and effective-mass renormalizations may emerge at low energies\cite{AvsarNatCommun2014,WangPhysRevX2016}.
In materials with strong SOC (\textit{e.g}., heavy-hole systems), higher-order variants such as $k^3$ Rashba terms produce anisotropic bands and additional spin-degeneracy points under illumination\cite{AvishaiPhysRevB2021,DellAnnaJPhysCondMatt2018}.
While that specific example pertains to heavy holes, the broader principle —that SOIR can markedly alter dispersion and transport—, remains relevant to graphene, especially under proximity to high-SOC substrates\cite{GindikinPhysRevB2025}. These dispersion changes can significantly impact the mobility and the density of states.
Notably, tuning SOIR in graphene can drive crossovers between Klein and anti-Klein tunneling across electrostatic barriers\cite{DellAnnaJPhysCondMatt2018}.
This control of barrier transmission \textit{via} SOIR underpins opportunities for Spintronics and quantum devices\cite{YaoMater2024}.

\section{State of the Art}\label{subsec:state-of-the-art}

\subsection{Related Publications}\label{subsec:RelPapers}
The theoretical foundations of fD tunneling in graphene have been extensively developed over the past decade. Early theoretical work established that in MLG, carriers preserve pseudospin alignment, leading to unit transmission at normal incidence regardless of barrier height\cite{Chen2016}. In BLG, the Berry phase differs, resulting in anti-Klein tunneling, where normal incidence yields zero transmission\cite{Du2018}. Further theoretical contributions revealed the potential for tuning between anti-Klein and Klein regimes in BLG by introducing a tunable bandgap \textit{via} perpendicular electric fields\cite{Du2018}. Opto-electronic analogies, such as negative refraction and Veselago lenses, were proposed, and models incorporating smooth \textit{versus} abrupt barriers, superlattices, and periodic scatterers were developed to describe how pseudospin conservation can be manipulated or suppressed to control tunneling\cite{Walls2015,An2020}.

\subsection{Experimental Validation}\label{subsec:Expepiriment}
Test verification of these theoretical predictions has provided compelling evidence for the unique transport properties of graphene. Opto-electronic experiments confirmed negative refraction and angular-dependent transmission using ballistic graphene p–n junctions in encapsulated devices\cite{Chen2016}.
Scanning-tunneling microscopy studies demonstrated quasi-bound states in nanometric graphene quantum dots, evidencing partial confinement of fD despite Klein tunneling\cite{Gutierrez2016}.
Visualization of electron flow through a Veselago lens using scanning gate microscopy provided direct imaging of electron focusing effects\cite{Brun2019}.
In BLG Fabry–Pérot interferometers, tuning the Fermi energy allowed observation of Berry phase evolution from $2\pi$ to $\pi$, accompanied by the transition from anti-Klein to Klein tunneling regimes\cite{Du2018}.
The clearest demonstration of both Klein and anti-Klein tunneling in a single experimental platform was achieved via a Corbino geometry device, establishing angle-resolved transmission characteristics consistent with theoretical predictions\cite{Elahi2024}. Device applications leveraging these fundamental phenomena have shown significant promise. Klein tunnel field-effect transistors exploit angular filtering using crossed p–n junctions to achieve current modulation without a bandgap\cite{Tan2017}.
Theoretical models forecast ON/OFF ratios above $10^4$ and steep subthreshold slopes under ideal conditions.
Experimental prototypes yielded ON/OFF ratios of approximately 10 and demonstrated current saturation and enhanced output resistance, indicating potential for high-frequency analog applications\cite{Wang2019}. Valley-selective Klein tunneling via superlattice barriers has been proposed as a route to valleytronic devices, though experimental realization remains pending\cite{An2020}.

\subsection{Critical Analysis and Opportunities}\label{subsec:future-work}
Despite significant advances in both theoretical understanding and experimental validation, several critical challenges and opportunities remain for future research in graphene-based quantum transport. Klein tunnel transistors exhibit limited ON/OFF contrast due to edge scattering; strategies such as edgeless geometries or improved gating precision could address this limitation. Valley-filtering mechanisms, while theoretically robust, lack experimental validation; designing patterned substrates or Moiré structures may provide a viable path forward. The exploration of trilayer graphene in terms of intermediate Berry phases and angular transmission profiles remains largely unexplored territory with significant potential. Integration of opto-electronic elements such as lenses and collimators in circuit-level architectures may unlock novel functional devices that leverage the unique properties of Dirac fermions. Investigating hybrid systems that combine tunneling control with superconductivity, strong magnetic fields, or Moiré engineering presents a particularly promising frontier for future technological developments. The transition from fundamental scientific curiosity to applied quantum engineering underscores graphene's remarkable adaptability and potential for transformative applications. Addressing current technical limitations while extending the established concepts to valleytronics and multilayer systems offers concrete research opportunities that can translate the unique physics of fD into revolutionary electronic and optoelectronic technologies. Issues such as: (i) the heavy-hole contribution to the SOIR effects for fD; (ii) the channel interference for fD fluxes impinging gate-biased quantum barriers in MLG nanoribbons, with or without SOC, remain yet insufficiently addressed. These limitations reinforce the need for robust predictive theoretical approaches, for the above mentioned physical situations. In this regard, the cornerstones of this theoretical exercise can be summarized as follows:

\begin{itemize}
    \item  Gives a clear analytical formulation of the probability-density current vector, whose mathematical expression explicitly considers the 
           channel interference, that becomes allowed for by how peudospinor terms couple, even at normal incidence of fD.
    \item  Develops a single theoretical framework that combines the channel interference together with the Rashba-type spin orbit interaction 
           effects on quantum scattering of fD in MLG. 
    \item  Offers a trustworthiness modeling of the quantum scattering phenomenology of fD; highlighting the role of several involved physical 
           quantities. Thus, with the aid of practical knobs with fundamental variables, such as the cross-coupling peudospinor terms in the probability-density current, we are able to trigger channel interference effects. On the other hand, based on workbench-electrostatic parameters, such as the SOIR strength (linear and cubic contributions) one can manage the fD tunneling probabilities.
    \item  Provides a systematic and comparative numerical analysis of quantum scattering coefficients for fD in pristine and under-SOIR MLG.
\end{itemize}

In the next section, the physical-mathematical model adopted in this study will be described in detail. Building on the conceptual foundation presented earlier regarding Quantum Scattering, CI and SOIR, the system and corresponding equations that allow numerical simulations of the phenomena observed in our results will be defined.



% Development
    \section{Theoretical Development}\label{sec:development}
    \input{development02}

% Discussion
    \section{Discussion of Results}\label{sec:discussion-of-results}
    \input{discussion02}

% Conclusion
    \section{Concluding Remarks}\label{sec:conclusions}

    In this work, we explored the effect of Rashba spin-orbit interaction (SOIR) on electronic transmission through a potential barrier in monolayer graphene using numerical simulations based on Gaussian wave packets and finite difference methods (FDM). The results revealed how the presence of the Rashba term introduces substantial and complex modifications to the transmission coefficients compared to the simple behavior observed in pristine graphene.
    These variations originate from spin-dependent scattering mechanisms, such as spin degeneracy lifting and quantum interferences derived from the multiple components of the wave packets used in our simulations.
    A key finding is the identification of channel interference effects between the pseudospinor components $\Psi_A$ and $\Psi_B$, which manifest as cross-coupling terms in the probability current density and fundamentally alter the transmission characteristics.

    Regarding the questions initially posed in this study, we successfully demonstrated that SOIR interactions do significantly modify transmission characteristics, thus affirmatively answering our initial inquiries.
    Additionally, we clarified that small deviations observed in pristine graphene from the theoretical unitary transmission are a direct consequence of the nature of the Gaussian packets employed, emphasizing the importance of considering these effects in simulations and practical applications.

    Collectively, our findings clearly highlight that a detailed understanding of SOIR provides predictive capability and essential tools for the engineering of spintronic devices and platforms applicable to quantum computing.
    Therefore, this study firmly reaffirms the relevance of these fundamental quantum phenomena in current and future technological scenarios. 
    
    Finally, we emphasize a key modeling choice: we neglect the canonical Dirac–Rashba term in pristine graphene because its magnitude is typically very small at the scales considered.
    Nevertheless, giant Rashba couplings have been reported in proximity-engineered systems (\textit{e.g.}, graphene on TMDs or with Au), for which the canonical term should be included explicitly and its impact on wave-packet propagation assessed.
    Extending our framework to that regime is straightforward (upgrade to a four-component spinor and add the Dirac–Rashba operator), and will be the subject of future work, where we will quantify the consequent spin splitting, spin precession, and possible Klein-to–anti-Klein transitions \cite{AvsarNatCommun2014, WangPhysRevX2016, DellAnnaJPhysCondMatt2018}.

% Acknowledgments
%\section*{Acknowledgments}
\ack{One of the authors (L.D-C) thanks for the support of DINVP and the hospitality of Departamento de Física y Matemáticas, Universidad Iberoamericana, Mexico City.}

\section*{Declaration on Generative AI}
\label{sec:genai}

This paper has been prepared with the assistance of several generative AI, for the sake of the grammar and clarity text improvements. Following the use of this tools, the authors thoroughly reviewed and edited the content as needed, and take full responsibility for the final version of the publication.

\bibliographystyle{unsrt}
\bibliography{bib/main}

\end{document}
