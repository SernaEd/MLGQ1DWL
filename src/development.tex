\subsection{Physical Model}\label{subsec:physical-model}

A quantum channel was fabricated using a hexagonal boron nitride substrate, upon which monolayer graphene (MLG) was deposited and subsequently capped with a silicon dioxide layer.
Two metallic electrodes were incorporated to generate a perpendicular electric field, thereby confining a Gaussian wave packet (GWP) within the graphene.
We also added a rectangular electrostatic potential barrier, where we incorporate the SOIR\@.
The system is more deeply explained in our previous article\cite{Serna2019}

% TODO: Add 3D image

\subsection{Mathematical Model}\label{subsec:mathematical-model}

From the physical model, we can get a mathematical model which describes the temporal evolution and consequently the quantum dispersion of the fDs in the MLG\@.

Starting with the pristine graphene hamiltonian\cite{Geimk2007}:

\begin{equation}
    \label{eq:pristGr}
    \hat{\bn{H}}_G = v_{\mb{\tiny F}}\vec{\mathbf{\sigma}}\cdot\vec{p},
\end{equation}

\noindent With $\vec{\mathbf{\sigma}} = \hat{\mathbf{\sigma}}_{x}\hat{\imath} + \hat{\mathbf{\sigma}}_{y}\hat{\jmath}$, being the pseudospin Pauli matrices $\hat{\mathbf{\sigma}}_{x} = \bigl(\begin{smallmatrix}
                                                                                                                                                                                                   0&1 \\ 1&0
\end{smallmatrix} \bigr)$, $\hat{\sigma}_{y} = \bigl(\begin{smallmatrix}
                                                         0&-i \\ i&0
\end{smallmatrix} \bigr)$, and $\vec{p}=\hat{p}_{x}\hat{\imath}+\hat{p}_{y}\hat{\jmath}$, the momentum operator, which $x, y$ components read $\hat{p}_{x} = -i\hbar\frac{\partial}{\partial x}$ and $\hat{p}_{y} = -i\hbar\frac{\partial}{\partial y}$, respectively.
From this point forward, $v_{\mb{\tiny F}}$ stands for the Fermi velocity of the carriers in MLG, which satisfies

\begin{equation}
    \label{eq:vF}
    v_{\mb{\tiny F}} \approx \frac{c}{300}.
\end{equation}

The previous hamiltonian can be rewritten as:

\begin{equation}
    \label{eq:pristineGrapheneMatrix}
    \hat{\bn{H}}_G = -i\hbar v_{\mb{\tiny F}}
    \begin{pmatrix}
        0                                                         & \frac{\partial}{\partial x}- i\frac{\partial}{\partial y} \\
        \frac{\partial}{\partial x}+ i\frac{\partial}{\partial y} & 0
    \end{pmatrix}.
\end{equation}

Assuming that the momentum-dependent term of the Rashba Hamiltonian for Q1D (quasi-one dimensional) semiconductor hetero-structures can be extended to the context of MLG-Q1D\cite{RDiago2010, Serna2019, RCDiagoEPL2015}, it follows that the hamiltonian for the SOIR can be written as:

\begin{equation}
    \label{eq:SOIRHamiltonian}
    \hat{\bn{H}}_R =
    \begin{pmatrix}
        0                        & k_{\alpha-} + k_{\beta-} \\
        k_{\alpha+} + k_{\beta+} & 0
    \end{pmatrix},
\end{equation}

\noindent Where $k_{\alpha\pm} = \alpha\left(k_{\pm}^2/k_{\mp}\right)$ and $k_{\beta\pm} = \beta k_{\pm}^3$; also defining that $k_{\pm}=k_x\pm i k_y$ which are the initial wave numbers of the system.
The symbols $\alpha$ and $\beta$ represent the linear and cubic contribution of the SOIR respectively.
%We are also taking into account that $\alpha = -\beta$, following the conclusions presented by Wong and Mireles\cite{WongUNAM2005}.
% TODO: Buscar alguna relación y/o valores para alfa y beta

For this research, we assume the heavy holes only in the region defined by the potential barrier.

The potential barrier is defined only in the specified region.

% TODO: Add potential barrier figure

With the following hamiltonian:

\begin{equation}
    \label{eq:potentialBarrierHamiltonian}
    \hat{\bn{H}}_V=\bn{I}_2 V(x)
\end{equation}

Adding up the graphene\eqref{eq:pristineGrapheneMatrix}, the SOIR\eqref{eq:SOIRHamiltonian}, and the potential barrier\eqref{eq:potentialBarrierHamiltonian} Hamiltonians, while also zeroing the parameters in the $y$ direction, we get:

\begin{equation}
    \label{eq:hybridHamiltonian}
    \hat{\bn{H}}=
    \begin{pmatrix}
        V(x)                                                                                   & \left(k_{\alpha-}+k_{\beta-}\right)-i\hbar v_{\mb{\tiny F}}\frac{\partial}{\partial x} \\
        \left(k_{\alpha+}+k_{\beta+}\right)-i\hbar v_{\mb{\tiny F}}\frac{\partial}{\partial x} & V(x)
    \end{pmatrix}.
\end{equation}

Note that if we wanted to see the results in the $y$ direction, we would have to zero the partial derivatives of $x$.

\subsection{Finite differences scheme}\label{subsec:finite-differences-scheme}

La principal ecuación a resolver es la ecuación de Schrödinger dependiente del tiempo:

\begin{equation}
    \label{eq:schrodingerTimeDependent}
    i\hbar \frac{d}{dt}\bn{\Psi}(x,t) = \hat{\bn{H}}\bn{\Psi}(x,t)
\end{equation}

En nuestro caso consideramos un paquete de ondas gausiano (GWP) para cada componente del pseudoespinor $\bn{\Psi}(x,t)=\begin{psmallmatrix*}
                                                                                                                           \psi_A(x,t)\\\psi_B(x,t)
\end{psmallmatrix*}$.
El GWP en un tiempo inicial tiene la siguiente forma:

\begin{equation}
    \label{eq:GWP}
    \psi_j(x,t_0)=\frac{\xi_j}{\sqrt[4]{\pi\sigma^2}}e^{-\frac{(x-x_0)^2}{2\sigma^2}}e^{ixk_0}
\end{equation}

\noindent Donde $\xi$ es la configuración inicial del pseudoespinor, $k_0$ es el número de onda inicial en $x$ y el subíndice $j$ indica la componente que se está usando.

Para el desarrollo de este artículo estamos ocupando las siguientes opciones de configuración del pseudoespinor:

\begin{equation}
    \label{eq:pseudospinorConfigurations}
    \xi=\begin{pmatrix}
            1\\0
    \end{pmatrix},\,\,\begin{pmatrix}
                          1\\1
    \end{pmatrix},\,\,\begin{pmatrix}
                          1\\i
    \end{pmatrix},\,\,\begin{pmatrix}
                          1\\e^{i\pi/4}
    \end{pmatrix}.
\end{equation}

El paquete de ondas entonces queda definido de la siguiente manera:

\begin{equation}
    \label{eq:FinalGWP}
    \bn{\Psi}(x,0)=\frac{1}{\sqrt {\xi_A^2 + \xi_B^2}}\begin{pmatrix}
                                                          \psi_A\\\psi_B
    \end{pmatrix}
\end{equation}

Para la densidad de probabilidad calculada a continuación, usamos la siguiente definición:

\begin{equation}
    \label{eq:probDens}
    \rho(x,t)=|\bn{\Psi}|^2 = |\psi_A|^2+|\psi_B|^2
\end{equation}

\subsection{Temporal Evolution Operator}\label{subsec:temporal-evolution-operator}
Para resolver la ecuación de Schrödinger\eqref{eq:schrodingerTimeDependent} se propone el uso de un operador de evolución temporal(TEO), a partir de la siguiente ecuación:

\begin{equation}
    \label{eq:TEOdifEq}
    \bn{\Psi}(x,t)=\hat{U}(t,t_0)\bn{\Psi}(x,t_0)
\end{equation}

\noindent Podemos ver que al aplicar el operador, obtenemos un tiempo $t$ para un tiempo $t_0$.
Si se sustituye la ecuación del TEO\eqref{eq:TEOdifEq} en la ecuación de Schrödinger\eqref{eq:schrodingerTimeDependent}, se puede desarrollar el álgebra y resolver la ecuación diferencial obteniendo al TEO\@.
Sabemos que el hamiltoniano no depende del tiempo, entonces podemos reescribir la ec.\eqref{eq:TEOdifEq} como:

\begin{equation}
    \label{eq:TEOApplied}
    \bn{\Psi}(x,t)= e^{-\frac{i\delta t}{\hbar}\hat{\bn{H}}} \bn{\Psi}(x,t_0)
\end{equation}

\noindent Donde $\delta t$ es el paso en el tiempo que discretizaremos más adelante.
La exponencial anterior se puede aproximar a través de la serie de Taylor de primer orden como:

\begin{equation}
    \label{eq:cayleyApprox}
    e^{-\frac{i\delta t}{\hbar}\hat{\bn{H}}} = \frac{2}{\bn{I} + \frac{i\delta t}{2\hbar}\hat{\bn{H}}}-bn{I}
\end{equation}

Haciendo uso de un cambio de variable podemos reescribir la ecuación anterior\eqref{eq:cayleyApprox} como:

\begin{equation}
    \label{eq:systemOfEquations}
    \bn{\Phi}(x,t_0) + \frac{i\delta t}{2\hbar}\hat{\bn{H}}\bn{\Phi}(x,t_0)=2\bn{\Psi}(x,t_0)
\end{equation}

\noindent Esta ecuación representa un sistema de ecuaciones de $2\times2$ (ya que tenemos dos incógnitas que son $\phi_A$ y $\phi_B$ y al desarrollar la matriz, podemos ver que se forman dos ecuaciones).

Como ambas ecuaciones tienen derivadas, ocupamos el método de diferencias finitas para resolver ambas ecuaciones para todo el espacio simultáneamente.
Para realizar esto, primero discretizamos al sistema de ecuaciones\eqref{eq:systemOfEquations} a lo largo de todo el espacio $j$ desde $0$ hasta $J$.

Para mantener la estabilidad del análisis numérico en el método de diferencias finitas, se debe respetar la siguiente condición\cite{Carrillo2015}:

\begin{equation}
    \label{eq:stabilityCondition}
    \delta t \leq \frac{\left( \delta x \right)^2}{2}
\end{equation}

Para discretizar las derivadas, ocupamos la expansión en serie de Taylor hasta el primer grado para el punto $x_j$ posterior y anterior:

\begin{align}
    \label{eq:TaylorBeforeAndAfter}
    f(x_j+\delta x)&=f(x_j) + (x_j+\delta x-x_j)f'(x_j)\nonumber\\
    &=f(x_j)+\delta xf'(x_j),\nonumber\\
    f(x_j-\delta x)&=f(x_j) + (x_j-\delta x-x_j)f'(x_j)\nonumber\\
    &=f(x_j)-\delta xf'(x_j),
\end{align}

\noindent si restamos la segunda ecuación de la primera y despejamos, podemos encontrar las ``Diferencias Centradas'' que podemos discretizar también como:

\begin{equation}
    \label{eq:diferenciasCentradas}
    f'(x_j)=\frac{f(x_j+\delta x)-f(x_j-\delta x)}{2\delta x} = \frac{f_{j+1}-f_{j-1}}{2\delta x}
\end{equation}

De este modo podemos reescribir nuestro sistema de ecuaciones, una vez realizada la discretización y el desarrollo algebraico de la matriz:

\begin{multline}
    \label{eq:partA}
    \phi_{A, j} + \frac{i\delta t}{2\hbar}(V_j\phi_{A,j} + \left(k_{\alpha-,j}+k_{\beta-,j}\right)\phi_{B,j}-\\
    i\hbar v_{\mb{\tiny F}}\frac{\phi_{B, j+1}-\phi_{B, j-1}}{2\delta x}) = 2\psi_{A,j}
\end{multline}

\begin{multline}
    \label{eq:partB}
    \phi_{B, j} + \frac{i\delta t}{2\hbar}(V_j\phi_{B,j} + \left(k_{\alpha+,j}+k_{\beta+,j}\right)\phi_{A,j}-\\
    i\hbar v_{\mb{\tiny F}}\frac{\phi_{A, j+1}-\phi_{A, j-1}}{2\delta x}) = 2\psi_{B,j}
\end{multline}

Si sumamos las ecuaciones ec.\eqref{eq:partA} y ec.\eqref{eq:partB}:

\begin{widetext}
    \begin{align}
        \label{eq:temporaryEquation}
        2\begin{pmatrix}
             \psi_{A,1} + \psi_{B,1} \\
             \psi_{A,2} + \psi_{B,2} \\
             \vdots                  \\
             \psi_{A,J} + \psi_{B,J} \\
        \end{pmatrix} &=
        \begin{pmatrix}
            N      & Q      & 0      & \ldots & \ldots & \ldots & 0 \\
            -Q     & N      & Q      & 0      & \ldots & \ldots & 0 \\
            0      & -Q     & N      & Q      & 0      & \ldots & 0 \\
            \vdots & 0      & -Q     & N      & Q      & \ddots & 0 \\
            \vdots & \vdots & 0      & -Q     & N      & \ddots & 0 \\
            \vdots & \vdots & \vdots & \ddots & \ddots & \ddots & Q \\
            0      & 0      & 0      & 0      & 0      & -Q     & N
        \end{pmatrix}\begin{pmatrix}
                         \phi_{A,1} \\
                         \phi_{A,2} \\
                         \vdots     \\
                         \phi_{A,J}
        \end{pmatrix}\\
        &+\begin{pmatrix}
              M      & Q      & 0      & \ldots & \ldots & \ldots & 0 \\
              -Q     & M      & Q      & 0      & \ldots & \ldots & 0 \\
              0      & -Q     & M      & Q      & 0      & \ldots & 0 \\
              \vdots & 0      & -Q     & M      & Q      & \ddots & 0 \\
              \vdots & \vdots & 0      & -Q     & M      & \ddots & 0 \\
              \vdots & \vdots & \vdots & \ddots & \ddots & \ddots & Q \\
              0      & 0      & 0      & 0      & 0      & -Q     & M
        \end{pmatrix}\begin{pmatrix}
                         \phi_{B,1} \\
                         \phi_{B,2} \\
                         \vdots     \\
                         \phi_{B,J}
        \end{pmatrix}
    \end{align}
\end{widetext}

\noindent Donde $N =1 + \frac{i\delta t}{2\hbar} \left( V_j + \left(k_{\alpha+,j}+k_{\beta+,j}\right)\right)$, $Q=\frac{i\hbar v_{\mb{\tiny F}}}{2\delta x}$ y $M =1 + \frac{i\delta t}{2\hbar}\left( V_j + \left(k_{\alpha-,j}+k_{\beta-,j}\right)\right)$.

Como se puede ver, el tamaño de la matriz depende del tamaño de nuestro sistema.
Si se escoge $\delta x = 1\angstrom$ entonces un pozo cuántico de $1200\angstrom$ generará una matriz de $1200\times1200$.

Repetimos el procedimiento anterior pero ahora restando las ecuaciones ec.\eqref{eq:partA} y ec.\eqref{eq:partB}

Viendo estas matrices podemos redefinirlas como dos ecuaciones matriciales sencillas $2\psi_+=\mathbb{N}\phi_A+\mathbb{M}\phi_B$ y $2\psi_-=\mathbb{L}\phi_A+\mathbb{P}\phi_B$.
Estas ecuaciones se pueden tratar como un sistema de ecuaciones matriciales que se resuelven así:

\begin{align}
    \label{eq:sistemaMatricial}
    \phi_A&=(2\mathbb{N}^{-1})\psi_+-(\mathbb{N}^{-1}\mathbb{M})\phi_B\nonumber\\
    \phi_B&=2(-\mathbb{L}\mathbb{N}^{-1}\mathbb{M}+\mathbb{P})^{-1}(\psi_--\mathbb{L}\mathbb{N}^{-1}\psi_+)
\end{align}

Una vez que contamos con el sistema de ecuaciones discretizado, se puede calcular la evolución temporal con:

\begin{equation}
    \label{eq:siguienteTiempo}
    \Psi_j^n+1=\Phi_j^n-\Psi_j^n
\end{equation}

Repitiendo el proceso para cada $n$ hasta llegar a $N$.

%TODO: Insertar una evolución temporal y describir las imagenes

\subsection{Probability Current Density}\label{subsec:probability-current-density}

Una vez que se cuenta con la evolución temporal, podemos calcular su coeficiente de transmisión a partir del cálculo de la corriente de densidad de probabilidad (PCD).
Para encontrar esta corriente, partimos desde la siguiente ecuación de continuidad:

\begin{equation}
    \label{eq:continuidad}
    \frac{\partial\rho}{\partial t} + \nabla\cdot\vec{j}=0
\end{equation}

Aplicando el traspuesto conjugado a la ecuación de Schrödinger dependiente del tiempo\eqref{eq:schrodingerTimeDependent}, y multiplicando por la derecha por $\Psi$; lo podemos incorporar la ecuación de la densidad de probabilidad\eqref{eq:probDens}.
Al mismo tiempo se considera el Hamiltoniano del grafeno\eqref{eq:pristGr}, el cual sabemos que es hermitiano ya que se compone de las matrices de Pauli.
De esta forma podemos construir la siguiente ecuación:

\begin{equation}
    \label{eq:rho_sigma_nabla}
    \frac{\partial}{\partial t}\int\rho d\vec{r} = -v_f\int\left( \Psi\vec{\sigma}\cdot\nabla\Psi^{T*} + \Psi^{T*}\vec{\sigma}\cdot\nabla\Psi \right)d\vec{r}
\end{equation}

Podemos reescribir la ecuación factorizando $\nabla$ y quitando la integral de ambos lados:

\begin{equation}
    \label{eq:casi_j}
    \frac{\partial}{\partial t}\rho = -v_f\nabla\cdot\left( \Psi^\dagger \vec{\sigma} \Psi \right)
\end{equation}

Sustituyendo en la ecuación de continuidad de la densidad de probabilidad ec.\eqref{eq:continuidad}:

\begin{equation}
    \label{eq:nablaJ}
    \nabla\cdot\vec{j} = v_f\nabla\cdot\left( \Psi^\dagger \vec{\sigma} \Psi \right)
\end{equation}

Eliminando $\nabla$ de ambos lados:

\begin{equation}
    \label{eq:final}
    \vec{j} = v_f\Psi^\dagger \vec{\sigma} \Psi
\end{equation}

De esta forma, hemos obtenido la densidad de corriente de probabilidad en grafeno.

Si queremos obtener las componentes de la densidad de corriente, podemos hacerlo de la siguiente forma:

\begin{equation}
    \label{eq:componentes}
    \vec{j} = v_f\begin{pmatrix}
                     \Psi^\dagger \sigma_x \Psi \\ \Psi^\dagger \sigma_y \Psi
    \end{pmatrix} = v_f\begin{pmatrix}
                           \Psi_A^\dagger\Psi_B + \Psi_B^\dagger\Psi_A \\ i(-\Psi_A^\dagger\Psi_B + \Psi_B^\dagger\Psi_A)
    \end{pmatrix}
\end{equation}

La importancia de la expresión del vector corriente, mostrada explícitamente en la ecuación\eqref{eq:componentes}, radica en la comprensión detallada del comportamiento de pseudoespines del grafeno.
La caracterización clara de ambas componentes del vector corriente resulta central para interpretar los resultados numéricos y entender la fenomenología asociada al transporte en estas estructuras.

A manera de comprobación, si consideramos una función de onda plana incidente, podemos expresarla como:

\begin{equation}
    \label{eq:ondaPlana}
    \Psi(\vec{r})=\frac{e^{i\vec{k}\cdot\vec{r}}}{\sqrt{2}}
    \begin{pmatrix}
        1 \\ e^{i\theta}
    \end{pmatrix},
\end{equation}

\noindent donde $\theta$ es el ángulo de incidencia respecto a la dirección normal a la barrera y $\vec{k} = k(\cos\theta,\sin\theta)$ es el vector de onda.

Si sustituimos la ec.\eqref{eq:ondaPlana} en la ec.\eqref{eq:componentes} podemos simplemente realizar la multiplicación de la siguiente forma para obtener la densidad de corriente de probabilidad incidente a la barrera sobre el eje $X$ ($j_{x,in}$):

\begin{align}
    \label{eq:jdemostrada}
    j_{x, in}&=v_f\frac{1}{2}
    \begin{pmatrix}
        e^{-i\theta} & 1
    \end{pmatrix}
    \begin{pmatrix}
        0 & 1 \\
        1 & 0
    \end{pmatrix}
    \begin{pmatrix}
        1 \\ e^{i\theta}
    \end{pmatrix}\nonumber \\
    &=v_f\left( \frac{e^{-i\theta} + e^{i\theta}}{2} \right)\nonumber \\
    &=v_f\cos\theta
\end{align}

Esta simplificación es lo que se usa habitualmente en la literatura\cite{DahalJPhysChemSolids2017, WuJAP2009}.

To determine the numerical value of $\vec{\jmath}$, we identify the peaks in the probability density plots.
Subsequently, we calculate the minima of the Gaussian curves to find the apparent width of the GWP. Finally, we locate two precise moments: immediately before the wave packet interacts with the potential barrier and the exact moment it has traversed the barrier.

Con los tiempos definidos de entrada ($t_1$) y de salida ($t_2$), usamos la ecuación ec.\eqref{eq:componentes} para todo el espacio en cada uno de los tiempos.

Con esto podemos ahora sí calcular el coeficiente de transmisión:

\begin{equation}
    \label{eq:transmissionCoef}
    T = \left| \frac{\vec{\jmath}_{out}\cdot\hat{n}}{\vec{\jmath}_{in}\cdot\hat{n}} \right|,
\end{equation}

\noindent En este contexto, se realiza el cociente entre la corriente transmitida y la corriente incidente, considerando el producto con el vector normal a la barrera.
En el caso unidimensional aquí abordado, dicho vector puede representarse como $(1,0)$ o $(0,1)$, dependiendo de si el paquete de onda gaussiano (GWP) se propaga a lo largo del eje $x$ o del eje $y$, respectivamente.

Una vez presentado el marco conceptual y las ecuaciones fundamentales del sistema estudiado, procedemos a continuación con el análisis y discusión crítica de los resultados obtenidos mediante las simulaciones numéricas.
Esta sección nos permitirá evaluar y contrastar detalladamente la influencia del término Rashba sobre la transmisión cuántica en grafeno monocapa.
