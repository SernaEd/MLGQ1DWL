% Introduction: Quantum Dispersion and Dirac Fermion Tunneling in Graphene

Graphene exhibits a linear dispersion near the Dirac points, leading to massless Dirac-fermion behavior.
The phenomenon of Klein tunneling—perfect transmission of carriers at normal incidence through high potential barriers—is a hallmark of such systems.
Conversely, bilayer graphene shows anti-Klein tunneling, i.e., total reflection at normal incidence in the absence of a bandgap.
Over the past decade, both theoretical predictions and experimental validations of these phenomena have matured significantly, particularly in monolayer and few-layer graphene systems.

In pristine monolayer graphene, the canonical (Dirac–Rashba) term is typically very small (often quoted in the $\mu$eV range), and thus it is commonly neglected for transport at the energy and length scales considered here.
This is the rationale behind not including of the standard Dirac–Rashba term explicitly in the pristine regions of our model.
However, the literature also reports proximity-induced scenarios where a giant Rashba coupling can emerge.
For example, in graphene interfaced with transition-metal dichalcogenides or in systems with Au adatoms/intercalation, leading to effective SOC values in the meV–tens of meV range\cite{AvsarNatCommun2014, WangPhysRevX2016}.

In such cases, the canonical Dirac–Rashba term should be included and its impact on wave-packet propagation analyzed (e.g., stronger spin splitting, spin precession, modified transmission resonances, and even transitions from Klein to anti-Klein regimes).
In the present study we focus on the pristine-graphene limit where the canonical term is negligible, while we model proximity-induced effects within the barrier via an effective Rashba-like coupling, as detailed in Sec.~\ref{subsec:mathematical-model}.

Finally, it is worth noting that one of the essential aspects addressed in this work lies in the precise analytical formulation of the current vector, whose analytical expression is given by equation\eqref{eq:componentes}.
This expression is the standard charge-current operator associated with the Dirac–Rashba model in graphene and has been discussed in prior literature\cite{AvishaiPhysRevB2021}.
Our contribution is to employ this established result to perform a component-wise and correlation-based analysis tailored to time-resolved Gaussian wave-packet transport under SOIR, which provides practical interpretability of the numerical transmission features reported in this study.


\subsection{Klein Paradox}\label{subsec:kleinParadox}

Applying an electrostatic potential to graphene can create regions where electrons are classically forbidden due to energy constraints\cite{SoninPhysRevB2009}.
For instance, a gate voltage can shift the local band structure (i.e., the Dirac point) relative to the Fermi level, creating a p-type region in which, at the incident electron energy, only valence-band states are available; classically, electrons lack sufficient energy to surmount this potential barrier\cite{DellAnnaJPhysCondMatt2018}.
However, in graphene, electron transmission through such barriers can exhibit unusual behavior due to the Klein paradox\cite{Young2009}.

The Klein paradox describes the phenomenon where electrons in graphene can tunnel through electrostatic potential barriers with a probability approaching unity, even at perpendicular incidence, seemingly defying the principles of classical mechanics\cite{TrauzettelNature2007}.
This counterintuitive transmission arises from the relativistic nature of electrons in graphene, where, within the barrier region, incident electrons can be converted into holes, allowing for unimpeded passage\cite{BernardiniJPhysAMathTheor2010}.
The Klein paradox is a direct consequence of graphene's unique linear dispersion relation and the associated massless Dirac fermion behavior.
While this perfect transmission through barriers might pose challenges for achieving electron confinement in some device designs, it also presents opportunities for developing novel tunneling-based electronic devices.
Beyond simply creating a forbidden region, the application of electrostatic potential can dramatically alter electron transmission through various other mechanisms, influencing the refractive index and leading to phenomena like electron lensing\cite{ParedesPhysRevB2021}.

\subsection{Rashba-Type Spin-Orbit Interaction}\label{subsec:rashba-type-spin-orbit-interaction}

SOIR is a relativistic effect that arises in systems lacking structural inversion symmetry\cite{AvishaiPhysRevB2021}.
SOIR is especially relevant due to its ability to significantly modify the electronic structure of graphene, inducing particular couplings between the spin and momentum of electrons.
This leads to emerging phenomena such as spin-dependent quantum interference and unconventional scattering effects.
The impact of SOIR not only provides new perspectives in fundamental physics but also opens promising possibilities for advanced applications in spintronic technologies and quantum information-based devices.

In graphene, SOIR can be induced by an external electric field applied perpendicular to the graphene plane or through proximity to a substrate\cite{ShcherbakovSciAdv2021}.
A key consequence of SOIR is the breaking of spin degeneracy, where the energy levels of electrons with opposite spins are no longer the same, even in the absence of an external magnetic field\cite{DelkhoshPhysE2015}.
Controlling spin degeneracy through electric fields via SOIR is particularly valuable for spintronic applications.
It enables electron spin manipulation without depending on magnetic fields, which typically require more power and are challenging to incorporate effectively into nanoscale systems.


Furthermore, SOIR can strongly reshape the band dispersion and spin textures in graphene and related heterostructures\cite{WangPhysRevX2016}.
In proximity-engineered systems that additionally break sublattice symmetry or combine multiple SOC channels, gaps and effective-mass renormalizations may emerge at low energies\cite{WangPhysRevX2016, AvsarNatCommun2014}.
In materials with strong spin–orbit coupling (e.g., heavy-hole systems), higher-order variants such as $k^3$ Rashba terms produce anisotropic bands and additional spin-degeneracy points under illumination\cite{DellAnnaJPhysCondMatt2018, AvishaiPhysRevB2021}.
While that specific example pertains to heavy holes, the broader principle—that SOIR can markedly alter dispersion and transport—remains relevant to graphene, especially under proximity to high-SOC substrates\cite{GindikinPhysRevB2025}.
These dispersion changes can significantly impact mobility and the density of states.
Notably, tuning SOIR in graphene can drive crossovers between Klein and anti-Klein tunneling across electrostatic barriers\cite{DellAnnaJPhysCondMatt2018}.
This control of barrier transmission via SOIR underpins opportunities for spintronic and quantum devices\cite{YaoMater2024}.

\subsection{State of the Art}\label{subsec:state-of-the-art}

The theoretical foundations of Dirac fermion tunneling in graphene have been extensively developed over the past decade.
Early theoretical work established that in monolayer graphene, carriers preserve pseudospin alignment, leading to unit transmission at normal incidence regardless of barrier height\cite{Chen2016}.
In bilayer graphene, the Berry phase differs, resulting in anti-Klein tunneling where normal incidence yields zero transmission\cite{Du2018}.
Further theoretical contributions revealed the potential for tuning between anti-Klein and Klein regimes in bilayer graphene by introducing a tunable bandgap via perpendicular electric fields\cite{Du2018}.
Electron-optics analogies, such as negative refraction and Veselago lenses, were proposed, and models incorporating smooth versus abrupt barriers, superlattices, and periodic scatterers were developed to describe how pseudospin conservation can be manipulated or suppressed to control tunneling\cite{Walls2015,An2020}.

Experimental verification of these theoretical predictions has provided compelling evidence for the unique transport properties of graphene.
Electron-optics experiments confirmed negative refraction and angular-dependent transmission using ballistic graphene p–n junctions in encapsulated devices\cite{Chen2016}.
STM studies demonstrated quasi-bound states in nanometric graphene quantum dots, evidencing partial confinement of Dirac fermions despite Klein tunneling\cite{Gutierrez2016}.
Visualization of electron flow through a Veselago lens using scanning gate microscopy provided direct imaging of electron focusing effects\cite{Brun2019}.
In bilayer graphene Fabry–Pérot interferometers, tuning the Fermi energy allowed observation of Berry phase evolution from $2\pi$ to $\pi$, accompanied by the transition from anti-Klein to Klein tunneling regimes\cite{Du2018}.
The clearest demonstration of both Klein and anti-Klein tunneling in a single experimental platform was achieved via a Corbino geometry device, establishing angle-resolved transmission characteristics consistent with theoretical predictions\cite{Elahi2024}.

Device applications leveraging these fundamental phenomena have shown significant promise.
Klein tunnel field-effect transistors (GKTFETs) exploit angular filtering using crossed p–n junctions to achieve current modulation without a bandgap\cite{Tan2017}.
Theoretical models forecast ON/OFF ratios above $10^4$ and steep subthreshold slopes under ideal conditions.
Experimental prototypes yielded ON/OFF ratios of approximately 10 and demonstrated current saturation and enhanced output resistance, indicating potential for high-frequency analog applications\cite{Wang2019}.
Valley-selective Klein tunneling via superlattice barriers has been proposed as a route to valleytronic devices, though experimental realization remains pending\cite{An2020}.


\subsection{Critical analysis and opportunities}\label{subsec:future-work}

Despite significant advances in both theoretical understanding and experimental validation, several critical challenges and opportunities remain for future research in graphene-based quantum transport.

Klein tunnel transistors exhibit limited ON/OFF contrast due to edge scattering; strategies such as edgeless geometries or improved gating precision could address this limitation.
Valley-filtering mechanisms, while theoretically robust, lack experimental validation; designing patterned substrates or moiré structures may provide a viable path forward.
The exploration of trilayer graphene in terms of intermediate Berry phases and angular transmission profiles remains largely unexplored territory with significant potential.

Integration of electron-optics elements such as lenses and collimators in circuit-level architectures may unlock novel functional devices that leverage the unique properties of Dirac fermions.
Investigating hybrid systems that combine tunneling control with superconductivity, strong magnetic fields, or moiré engineering presents a particularly promising frontier for future technological developments.

The transition from fundamental scientific curiosity to applied quantum engineering underscores graphene's remarkable adaptability and potential for transformative applications.
Addressing current technical limitations while extending the established concepts to valleytronics and multilayer systems offers concrete research opportunities that can translate the unique physics of Dirac fermions into revolutionary electronic and optoelectronic technologies.

In the next section, the physical-mathematical model adopted in this study will be described in detail.
Building on the conceptual foundation presented earlier regarding quantum dispersion and Rashba-type spin-orbit interaction, the system and corresponding equations that allow numerical simulation of the phenomena observed in our results will be defined.

